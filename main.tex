\documentclass[12pt,openany]{book}
\usepackage{graphicx,multirow,amsmath,enumitem,amssymb,tabularx,subcaption,siunitx,fancyhdr,diagbox,tikz,hyperref,amsthm,appendix,microtype,xcolor,color,pifont}
\usepackage[a4paper,left=20mm,right=20mm,top=25mm,bottom=25mm]{geometry}
\theoremstyle{definition}
\newtheorem{theorem}{Theorem}[section]
\newtheorem{lemma}[theorem]{Lemma}
\newtheorem{definition}[theorem]{Definition}
\newtheorem{example}[theorem]{Example}
\newtheorem{proposition}[theorem]{Proposition}
\newtheorem{corollary}[theorem]{Corollary}
\newcommand{\p}[0]{\mathbb{P}}
\title{Lecture Notes for Harmonic Analysis}
\author{Yitong Qiu\\
School of the Gifted Young\\
University of Science and Technology of China}
\date{Spring 2024}

\begin{document}

\maketitle
\tableofcontents
\setcounter{page}{0}
\newpage

\chapter{Introduction}
\section{Supplement knowledge}
\noindent\textbf{(Interpolation)} Let $f$ be a measurable function on the measure space $(X, M, \mu)$. The distribution function of $f$ is defined as $d_f: [0, \infty) \rightarrow [0, \infty]$ given by

$$d_f(\lambda) = \mu(\{x \in X: |f(x)| > \lambda\})$$as a function of $\lambda$.

\noindent\textbf{(Weak $L^p$ norm)} For $0 < p < \infty$, the weak $L^p$ norm of $f$ is defined as

$$\|f\|_{p, \infty} = \sup_{\lambda > 0} \left(\lambda^p \mu(\{x \in X: |f(x)| > \lambda\})\right)^{\frac{1}{p}}.
$$
\noindent\textbf{(Weak $L^p$ space)} The weak $L^p$ space, denoted by $L^{p, \infty}$, is the set of all functions $f$ such that $\|f\|_{p, \infty} < \infty$.

\noindent\textbf{Remark:} The weak $L^{\infty}$ space is identical to the standard $L^{\infty}$ space.

\begin{proposition}
    If $0 < p < \infty$, then

$$\|f\|_p^p = \int |f|^p d\mu = p \int_0^{\infty} \alpha^{p-1} d_f(\alpha) d\alpha.$$
\end{proposition} 
\begin{proof}
    Assuming $\mu$ is a suitable measure, we have
$$
\begin{aligned}
p \int_0^{\infty} \alpha^{p-1} \mu(\{x \in X: |f(x)| > \alpha\}) d\alpha &= p \int_0^{\infty} \alpha^{p-1} \int_X \chi_{\{x: |f(x)| > \alpha\}} d\mu d\alpha \\
&\stackrel{\text{Fubini}}{=} \int_X \int_0^{|f(x)|} p \alpha^{p-1} d\alpha d\mu(x) \\
&= \int_X |f(x)|^p d\mu(x) = \|f\|_p^p.
\end{aligned}
$$
\end{proof}
\begin{theorem}[The Marcinkiewicz Interpolation Theorem]
    Let $(X, \mu)$ and $(Y, \nu)$ be measure spaces. Suppose that $p_0, p_1, q_0, q_1$ are elements of $[1, \infty]$ satisfying $p_0 \leqslant q_0$, $p_1 \leqslant q_1$, and $q_0 \neq q_1$. Consider $T$ to be a sublinear map ($|T(cf)| = c|T(f)|$ and $|T(f+g)| \leqslant |T(f)| + |T(g)|$) from $L^{p_0}(\mu) + L^{p_1}(\mu)$ to the space of measurable functions on $Y$. If $T$ is of weak type $(p_0, q_0)$ and $(p_1, q_1)$, then for any $0 < t < 1$ and $\left(\frac{1}{p}, \frac{1}{q}\right) = (1-t)\left(\frac{1}{p_0}, \frac{1}{q_0}\right) + t\left(\frac{1}{p_1}, \frac{1}{q_1}\right)$, the map $T$ is of strong type $(p, q)$.Then, $B_p$ remains bounded as $p \rightarrow p_j$ if $p_j < \infty$ (or as $p \rightarrow \infty$ if $p_j = \infty$).
\end{theorem}
\noindent\textbf{Remark:}
Weak type means that there exist constants $C_0$ and $C_1$ such that $\|Tf\|_{L^{q_0, \infty}} \leqslant C_0\|f\|_{L^{p_0}}$ and $\|Tf\|_{L^{q_1, \infty}} \leqslant C_1\|f\|_{L^{p_1}}$.
Strong type means that there exists a constant $B_p$ depending only on $p, p_j, q_j, C_j$ (for $j = 0, 1$) such that $\|Tf\|_{L^q} \leqslant B_p\|f\|_{L^p}$. 
\begin{theorem}[The Riesz-Thorin Interpolation Theorem]
    
Let $(X, \mu)$ and $(Y, \nu)$ be two measure spaces, and let $1 \leqslant p_0, p_1, q_0, q_1 \leqslant \infty$. If $q_0 = q_1 = \infty$, assume further that $\nu$ is semifinite. Let $T$ be a linear operator from $L^{p_0}(X) + L^{p_1}(X)$ into $L^{q_0}(Y) + L^{q_1}(Y)$ such that:

1. $\|Tf\|_{L^{q_0}} \leqslant M_0\|f\|_{L^{p_0}}$ for all $f \in L^{p_0}$ (strong type $(p_0, q_0)$).

2. $\|Tf\|_{L^{q_1}} \leqslant M_1\|f\|_{L^{p_1}}$ for all $f \in L^{p_1}$ (strong type $(p_1, q_1)$).

Then, for any $0 < \theta < 1$ and $\left(\frac{1}{p}, \frac{1}{q}\right) = (1-\theta)\left(\frac{1}{p_0}, \frac{1}{q_0}\right) + \theta\left(\frac{1}{p_1}, \frac{1}{q_1}\right)$, we have $\|Tf\|_{L^q} \leqslant M_0^{1-\theta}M_1^\theta\|f\|_{L^p}$ for all $f \in L^p$ (strong type $(p, q)$).
\end{theorem}
\section{Maximal functions}
$f \in L_{loc}^1(\mathbb{R}^n)$ The Hardy-Littlewood maximal function of $f$ is defined as
$$
M f(x) = \sup_{r > 0} \frac{1}{|B_r|} \int_{B_r} |f(x - y)| \, dy \quad \text{where} \quad B_r = B(0, r) \subset \mathbb{R}^n
$$
which can also be expressed as
$$
M f(x) = \sup_{r > 0} \frac{1}{|B_r|} \int_{B(x, r)} |f(y)| \, dy
$$
or equivalently as
$$
M f(x) = \sup_{r > 0} \frac{1}{|B_r|} (\chi_{B_r} * |f|)(x)
$$
where $\chi_{B_r}$ is the characteristic function of the ball $B_r$.

An alternative definition is given by
$$
\tilde{M} f(x) = \sup_{x \in B} \frac{1}{|B|} \int_B |f(y)| \, dy
$$
where the supremum is taken over all balls $B$ containing $x$. It can be shown that $M f(x)$ and $\tilde{M} f(x)$ are equivalent in the sense that there exists a constant $C > 0$ such that
$$
M f(x) \leq \tilde{M} f(x) \leq C M f(x)
$$
denoted as $\tilde{M} f(x) \lesssim M f(x)$.
\begin{lemma}[A finite version of the Vitali covering lemma)]
    Let $E$ be a measurable subset of $\mathbb{R}^n$ that is the union of a finite collection of balls $\{B_j\}$. Then one can select a disjoint subcollection $B_1, \ldots, B_m$ of the $\{B_j\}$ so that
$$
\sum_{k=1}^m |B_k| > C |E|
$$
with $C = 3^{-n}$.
\end{lemma}
\begin{theorem}
    Let $f$ be a function defined on $\mathbb{R}^n$.
(a) The operator $M$ is of weak type $(1, 1)$, i.e., if $f \in L^1$, $\lambda > 0$, then
$$
m(\{x : |M f(x)| > \lambda\}) \leq \frac{C}{\lambda} \|f\|_1
$$
where $m$ denotes the Lebesgue measure.

(b) $M$ is of strong type $(p, p)$ for $1 < p \leq \infty$, i.e.,
$$
\|M f\|_p \leq A_p \|f\|_p
$$
where $A_p$ is a constant depending only on $p$ and $n$.
\end{theorem}
\noindent\textbf{Remark:} If $f$ is in $L^1$ and is not identically zero, then $M f \notin L^\infty$.

Since $f$ is not identically zero, there exist $\varepsilon > 0$ and $R > 0$ such that
$$
\int_{B_R} |f| \geq \varepsilon > 0
$$
If $|x| > R$, then $B_R \subset B(x, 2|x|)$ and
$$
M f(x) \geq \frac{1}{|B(x, 2|x|)|} \int_{B(x, 2|x|)} |f| \, dv \geq \frac{C}{|x|^n} \int_{B_R} |f| \, dv \geq \frac{C \varepsilon}{|x|^n}
$$
which is not in $L^1$. Therefore, $M f \notin L^\infty$.

\begin{proof}
    Obviously, $\|M f\|_\infty \leq \|f\|_\infty$, so by the Marcinkiewicz interpolation theorem, it suffices to prove (a).
Let $E_\lambda = \{x : |\tilde{M} f(x)| > \lambda\}$. For all $x \in E_\lambda$, there exists a ball $B_x$ such that
$$
\frac{1}{m(B_x)} \int_{B_x} |f(y)| \, dy > \lambda
$$
Fix a compact subset $K$ of $E_\lambda$. Since $K$ is covered by $\bigcup_{x \in E_\lambda} B_x$, we can select a finite subcover $K \subset \bigcup_{l=1}^N B_l$. By the covering lemma, there exist disjoint balls $B_{i1}, \ldots, B_{ik}$ such that
$$
m\left(\bigcup_{l=1}^N B_l\right) \leq C \sum_{j=1}^k m(B_{ij})
$$
Therefore,
$$
m(K) \leq m\left(\bigcup_{l=1}^N B_l\right) \leq C \sum_{j=1}^k m(B_{ij}) \leq \frac{C}{\lambda} \sum_{j=1}^k \int_{B_{ij}} |f| \, dy = \frac{C}{\lambda} \int_{\bigcup B_{ij}} |f| \, dy \leq \frac{C}{\lambda} \|f\|_1
$$
Letting $m(K) \to m(E_\lambda)$ completes the proof of (a).
\end{proof} 

As a corollary, we obtain the Lebesgue differentiation theorem:
\begin{theorem}
    If $f \in L_{loc}^1(\mathbb{R}^n)$, then
$$
\lim_{r \to 0} \frac{1}{|B(x, r)|} \int_{B(x, r)} f(y) \, dy = f(x) \quad \text{for a.e.} \quad x
$$
\end{theorem}
\section{Approximation to the identity.}
\begin{definition}
    Let $\varphi$ be a function belonging to $L^{\prime}\left(\mathbb{R}^n\right)$, the dual space of Lebesgue integrable functions on $\mathbb{R}^n$, satisfying $\int_{\mathbb{R}^n} \varphi(x) \, dx = 1$. Define $\varphi_{\varepsilon}(x) = \varepsilon^{-n} \varphi\left(\frac{x}{\varepsilon}\right)$ for $\varepsilon > 0$. The family of functions $\left\{\varphi_{\varepsilon}\right\}_{\varepsilon > 0}$ is known as an approximation to the identity.
\end{definition}
\begin{theorem}
    Suppose $\left\{\varphi_{\varepsilon}\right\}_{\varepsilon > 0}$ is an approximation to the identity. Then for any function $f$ in $L^p(\mathbb{R}^n)$ with $1 \leq p < \infty$, we have

$$\lim_{\varepsilon \to 0} \left\|\varphi_{\varepsilon} * f - f\right\|_p = 0$$where $*$ denotes convolution.
\end{theorem}
\noindent\textbf{Remark:}There exists a sequence $\{\varepsilon_k\}_{k \in \mathbb{N}}$ converging to 0 as $k \to \infty$ such that $\varphi_{\varepsilon_k} * f(x) = f(x)$ almost everywhere.
\begin{proof}
    Since $\int_{\mathbb{R}^n} \varphi(x) \, dx = 1$, we can write

$$\varphi_{\varepsilon} * f(x) - f(x) = \int_{\mathbb{R}^n} \varphi(y) [f(x - \varepsilon y) - f(x)] \, dy$$Given $\tilde{\varepsilon} > 0$, choose $\delta > 0$ such that if $|h| < \delta$, then $\|f(t + h) - f(t)\| < \frac{\tilde{\varepsilon}}{2\|\varphi\|_1}$. For this fixed $\delta$, if $\varepsilon$ is sufficiently small, then

$$\int_{|y| \geq \frac{\delta}{\varepsilon}} |\varphi(y)| \, dy \leq \frac{\tilde{\varepsilon}}{4\|f\|_p}$$Using Minkowski's integral inequality, we have

$$\begin{aligned}
\left\|\varphi_{\varepsilon} * f - f\right\|_p &\leq \int_{\mathbb{R}^n} |\varphi(y)| \|f(x - \varepsilon y) - f(x)\|_{L^p} \, dy \\
&< \int_{|y| < \frac{\delta}{\varepsilon}} |\varphi(y)| \|f(x - \varepsilon y) - f(x)\|_{L^p} \, dy + 2\|f\|_p \int_{|y| \geq \frac{\delta}{\varepsilon}} |\varphi(y)| \, dy \\
&\leq \frac{\tilde{\varepsilon}}{2\|\varphi\|_1} \cdot \|\varphi\|_1 + 2\|f\|_p \cdot \frac{\tilde{\varepsilon}}{4\|f\|_p} = \tilde{\varepsilon}
\end{aligned}$$
\end{proof}
\begin{definition}[Schwartz Space $S(\mathbb{R}^n)$]
    The Schwartz space $S(\mathbb{R}^n)$ is defined as the set of all infinitely differentiable functions $f$ on $\mathbb{R}^n$ such that for any multi-indices $\alpha$ and $\beta$, the supremum
$\sup_{x \in \mathbb{R}^n} |x^\alpha D^\beta f(x)|$
is finite. Here, $x^\alpha = x_1^{\alpha_1} x_2^{\alpha_2} \cdots x_n^{\alpha_n}$ and $D^\beta = \partial_{x_1}^{\beta_1} \partial_{x_2}^{\beta_2} \cdots \partial_{x_n}^{\beta_n}$, where $x = (x_1, x_2, \ldots, x_n) \in \mathbb{R}^n$ and $\alpha, \beta \in \mathbb{N}^n$.
\end{definition}
As an example, the function $f(x) = e^{-|x|^2}$ belongs to $S(\mathbb{R}^n)$.
If $f$ belongs to the Schwartz space $S$, then for any multi-index $\alpha$, the function $D^\alpha(x^\beta f)$ also belongs to $S$.
\begin{proposition}
    If $f \in S(\mathbb{R}^n)$, then for any multi-index $\beta$ and any natural number $N$, there exists a constant $C_{N, \beta}$ such that
$|D^\beta f(x)| \leqslant \frac{C_{N, \beta}}{(1 + |x|)^N}.$
As a consequence, $D^\beta f$ belongs to $L^p$ for all $p \geqslant 1$.
\end{proposition}
\begin{proposition}
Let $\varphi \in L^{\prime}\left(\mathbb{R}^n\right)$ with $\int \varphi = 1$, and define $\varphi_{\varepsilon}(x) = \varepsilon^{-n} \varphi\left(\varepsilon^{-1} x\right)$. Then, for all $f \in S\left(\mathbb{R}^n\right)$,

$$\lim_{\varepsilon \rightarrow 0} (f * \varphi_{\varepsilon})(x) = f(x).$$
\end{proposition}
\begin{proof}
     Consider the convolution $f * \varphi_{\varepsilon}(x) = \int f(x - \varepsilon y) \varphi(y) dy$. We have the estimate

$$|f(x - \varepsilon y) \varphi(y)| \leqslant \|f\|_{\infty}|\varphi(y)|,$$
which belongs to $L^{\prime}$ since $\varphi \in L^{\prime}\left(\mathbb{R}^n\right)$ and $f$ is bounded. Therefore, by the Dominated Convergence Theorem (DCT),
$$\lim_{\varepsilon \rightarrow 0} (f * \varphi_{\varepsilon})(x) = \int f(x) \varphi(y) dy = f(x) \int \varphi(y) dy = f(x),$$
as desired.
\end{proof}
\begin{theorem}
    Let $\varphi \in L^{1}$ with $\int \varphi=1$, and define $\varphi_{\varepsilon}(x)=\varepsilon^{-n} \varphi(\varepsilon^{-1} x)$ and $\psi(x)=\sup _{|y| \geqslant|x|}|\varphi(y)|$ (the least decreasing radial majorant of $\varphi$). Suppose that $\psi \in L^{1}$. Then,

(1) For all $f \in L^p(\mathbb{R}^n)$ with $1 \leqslant p \leqslant \infty$, we have $\sup_{\varepsilon>0}|f * \varphi_{\varepsilon}(x)| \leqslant A Mf(x)$ almost everywhere, where $A=\int_{\mathbb{R}^n} \psi(x) dx$ and $Mf$ is the Hardy-Littlewood maximal function of $f$.

(2) For all $f \in L^p$ with $1 \leqslant p < \infty$, we have $\lim_{\varepsilon \rightarrow 0} f * \varphi_{\varepsilon}(x)=f(x)$ almost everywhere. (This is known as differentiation of the approximation to identity.)
\end{theorem}
\begin{proof}
    


(1) It suffices to show that $\sup_{\varepsilon>0}|f_1 * \psi_{\varepsilon}(x)| \leqslant \|\psi\|_1 Mf(x)$ for any radial decreasing function $\psi \in L^1(\mathbb{R}^n)$ (i.e., $\psi(x) \leqslant \psi(y)$ if $|x| > |y|$).

\textbf{Step 1:} Assume first that $\psi$ is a simple function, say $\psi(x)=\sum_{j=1}^m b_j \chi_{R_j}(x)$ where $\{R_j\}$ are annuli centered at the origin and $\{b_j\}$ are distinct positive numbers with $b_1 > \cdots > b_m$. We can write

$$\psi(x) = \sum_{j=1}^m (b_j - b_{j+1}) \chi_{B_j}(x)$$where $B_j$ are concentric balls centered at the origin, $B_j = \{x: |x| \leqslant r_j\}$ for some $r_j$, and $b_{m+1}=0$. Let $a_j = b_j - b_{j+1} > 0$ for $j=1,\ldots,m-1$ and $a_m = b_m > 0$. Then,

$$\int \psi(x) dx = \sum_{j=1}^m a_j |B_j|.$$Now,

$$|f| * \psi(x) = \int |f|(x-y) \psi(y) dy = \sum_{j=1}^m a_j \int_{B_j} |f|(x-y) dy \leqslant Mf(x) \sum_{j=1}^m a_j |B_j| = \|\psi\|_1 Mf(x).$$

\textbf{Step 2:} For a general radial decreasing function $\psi \in L^1(\mathbb{R}^n)$, there exists a sequence of simple functions $\{\psi_k\}$ as in Step 1 such that $0 \leqslant \psi_1 \leqslant \cdots \leqslant \psi_k \leqslant \cdots$ and $\psi_k \leqslant \psi$ for all $k \in \mathbb{N}$, with $\psi(x) = \lim_{k \rightarrow \infty} \psi_k(x)$. Clearly, $\|\psi_k\|_1 \leqslant \|\psi\|_1$ for all $k$. By Fatou's Lemma, we have

$$\begin{aligned}
    |f| * \psi(x) = \int |f|(x-y) \lim_{k \rightarrow \infty} \psi_k(y) dy &\leqslant \liminf_{k \rightarrow \infty} \int |f|(x-y) \psi_k(y) dy \\
    &\leqslant \liminf_{k \rightarrow \infty} \|\psi_k\|_1 Mf(x)\\
    &\leqslant \|\psi\|_1 Mf(x).
\end{aligned}$$

\textbf{Step 3:} Replace $\psi$ in Step 2 by $\psi_{\varepsilon}$ to get

$$|f| * \psi_{\varepsilon}(x) \leqslant \|\psi_{\varepsilon}\|_1 Mf(x) = \|\psi\|_1 Mf(x).$$Taking the supremum over $\varepsilon > 0$ completes the proof of part (1).

(2) For $1 \leqslant p < \infty$, it is known that $\lim_{\varepsilon \rightarrow 0} \|f * \varphi_{\varepsilon} - f\|_p = 0$, which implies that $\lim_{\varepsilon_k \rightarrow 0} f * \varphi_{\varepsilon_k}(x) = f(x)$ almost everywhere for any sequence $\{\varepsilon_k\}$ converging to 0. It remains to show that $\lim_{\varepsilon \rightarrow 0} f * \varphi_{\varepsilon}(x)$ exists almost everywhere.

Define $\Omega f(x) = |\limsup_{\varepsilon \rightarrow 0} f * \varphi_{\varepsilon}(x) - \liminf_{\varepsilon \rightarrow 0} f * \varphi_{\varepsilon}(x)|$. If $f \in S(\mathbb{R}^n)$ (the Schwartz space), then $\lim_{\varepsilon \rightarrow 0} f * \varphi_{\varepsilon}(x) = f(x)$ for all $x \in \mathbb{R}^n$, so $\Omega f(x) \equiv 0$. For a general $f \in L^p$ with $1 \leqslant p < \infty$, we can write $f = f_1 + f_2$ where $f_1 \in S(\mathbb{R}^n)$ and $\|f_2\|_p$ is arbitrarily small,since the Schwartz space is densely embedded in the $L^p$ space. Then,

$$\Omega f(x) \leqslant \Omega f_1(x) + \Omega f_2(x) = \Omega f_2(x) \leqslant 2A Mf_2(x).$$Therefore,

$$|\{x: \Omega f(x) \geqslant \varepsilon\}| \leqslant |\{x: Mf_2(x) > \frac{\varepsilon}{2A}\}| \leqslant C \left(\frac{\|f_2\|_p}{\varepsilon / 2A}\right)^p.$$Since $\|f_2\|_p$ can be made arbitrarily small, we conclude that $|\{x: \Omega f(x) > \varepsilon\}| = 0$ for all $\varepsilon > 0$, and hence $|\{x: \Omega f(x) > 0\}| = 0$. This completes the proof of part (2) for $1 \leqslant p < \infty$.

For $p = \infty$, fix any ball $B$ and let $B_1$ be a larger ball containing $B$ with $\delta$ the distance from $B$ to $\mathbb{R}^n \setminus B_1$. Write $f = f_1 + f_2$ where $f_1 = f \chi_{B_1}$ and $f_2 = f - f_1$. Then $f_1 \in L^1$ and for $x \in B$, we have

$$|f_2 * \varphi_{\varepsilon}(x)| = \left|\int f_2(x-y) \varphi_{\varepsilon}(y) dy\right| \leqslant \|f\|_{\infty} \int_{|y| \geqslant \delta} |\varphi_{\varepsilon}(y)| dy = \|f\|_{\infty} \int_{|y| \geqslant \delta/\varepsilon} |\varphi(y)| dy \rightarrow 0$$as $\varepsilon \rightarrow 0$. Since $\lim_{\varepsilon \rightarrow 0} f_1 * \varphi_{\varepsilon}(x) = f_1(x)$ almost everywhere, we conclude that $\lim_{\varepsilon \rightarrow 0} (f_1 + f_2) * \varphi_{\varepsilon}(x) = f_1(x) = f(x)$ for almost every $x \in B$. This completes the proof of part (2) for $p = \infty$.
\end{proof}
\begin{theorem}
    Let $\left\{T_\varepsilon\right\}$ be a family of linear operators mapping from $L^p\left(\mathbb{R}^n\right)$ into the space of measurable functions from $\mathbb{R}^n$ to $\mathbb{C}$. Define the operator $T^*$ as $T^* f(x)=\sup _{\varepsilon>0}\left|T_{\varepsilon} f(x)\right|$. If $T^*$ is of weak type $(p, q)$, then the set $\left\{f \in L^p\left(\mathbb{R}^n\right): \lim _{\varepsilon \rightarrow \varepsilon_0} T_{\varepsilon} f(x)=f(x) \text{ almost everywhere}\right\}$ is closed in $L^p\left(\mathbb{R}^n\right)$.
\end{theorem}
\noindent\textbf{Remark}: $T^*$ is the maximal operator associated with the family $\left\{T_{\varepsilon}\right\}$.
\begin{proof}
    Consider a sequence of functions $\left\{f_n\right\} \subset L^p$ converging to $f$ in $L^p$ and satisfying $T_{\varepsilon} f_n(x) \rightarrow f_n(x)$ almost everywhere. Then,

$$\begin{aligned}
& m\left(\left\{x: \limsup _{\varepsilon \rightarrow \varepsilon_0}\left|T_{\varepsilon} f(x)-f(x)\right|>\lambda\right\}\right) \\
& \leqslant m\left(\left\{x:\limsup _{\varepsilon \rightarrow \varepsilon_0}\left|T_{\varepsilon}\left(f-f_n\right)(x)-\left(f-f_n\right)(x)\right|>\lambda\right\}\right) \\
& \leqslant m\left(\left\{x:\left|T^*\left(f-f_n\right)(x)\right|>\frac{\lambda}{2}\right\}\right) + m\left(\left\{x:\left|\left(f-f_n\right)(x)\right|>\frac{\lambda}{2}\right\}\right) \\
& \leqslant \left(\frac{2 C}{\lambda}\left\|f-f_n\right\|_p\right)^q + \left(\frac{2}{\lambda}\left\|f-f_n\right\|_p\right)^p \rightarrow 0 \quad \text{as } n \rightarrow \infty.
\end{aligned}$$
Therefore,
$$m\left(\left\{x: \limsup _{\varepsilon \rightarrow \varepsilon_0}\left|T_{\varepsilon} f(x)-f(x)\right|>\lambda\right\}\right) = 0.$$
Hence,
$$m\left(\left\{x: \limsup _{\varepsilon \rightarrow \varepsilon_0}\left|T_{\varepsilon} f(x)-f(x)\right|>0\right\}\right) \leqslant \sum_{k=1}^{\infty} m\left(\left\{x: \limsup _{\varepsilon \rightarrow \varepsilon_0}\left|T_{\varepsilon} f(x)-f(x)\right|>\frac{1}{k}\right\}\right) = 0.$$
Thus, $f$ belongs to the given set.
\end{proof}

\begin{theorem}
  If $|\phi(x)| \leqslant \psi(x)$ almost everywhere, where $\psi$ is a non-negative, radially decreasing, and integrable function, and $f \in L^p\left(\mathbb{R}^n\right)$ for $1 \leq p < \infty$, then

$$\lim _{\varepsilon \rightarrow 0} \phi_{\varepsilon} * f(x) = \left(\int \phi\right) f(x)$$

almost everywhere.  
\end{theorem}
\begin{proof}
    By some known results, $\sup _{\varepsilon>0}\left|\phi_{\varepsilon} * f(x)\right|$ is of weak type $(1,1)$ and strong type $(p, p)$ for $1 < p < \infty$.
From the previous theorem, the set

$$\left\{f \in L^p\left(\mathbb{R}^n\right): \lim _{\varepsilon \rightarrow 0} \phi_{\varepsilon} * f(x) = \left(\int \phi\right) f(x) \text{ a.e.}\right\}$$

is closed in $L^p\left(\mathbb{R}^n\right)$.
Since $S \subset \left\{f \in L^p\left(\mathbb{R}^n\right): \lim _{\varepsilon \rightarrow 0} \phi_{\varepsilon} * f(x) = \left(\int \phi\right) f(x) \text{ a.e.}\right\} \subset L^p$,
taking the closure, we have $\bar{S} = L^p$.
Therefore,

$$\left\{f \in L^p\left(\mathbb{R}^n\right): \lim _{\varepsilon \rightarrow 0} \phi_{\varepsilon} * f(x) = \left(\int \phi\right) f(x) \text{ a.e.}\right\} = L^p.$$
\end{proof}
\begin{example}
    Let $P(x) = \frac{C_n}{\left(1 + |x|^2\right)^{\frac{n+1}{2}}}$ be the Poisson kernel, where $C_n = \frac{\Gamma\left(\frac{n+1}{2}\right)}{\pi^{\frac{n+1}{2}}}$.
Define

$$P_t(x) = C_n \frac{t}{\left(t^2 + |x|^2\right)^{\frac{n+1}{2}}}.$$

Now, let $u(x, t) = P_t(x) * f(x)$. Then, for $f \in L^p$, the function $u(x, t)$ solves the Dirichlet problem

$$\begin{cases}
\left(\Delta_x + \frac{\partial^2}{\partial t^2}\right) u = 0 & \text{in } \mathbb{R}_{+}^{n+1} = \left\{(x, t): x \in \mathbb{R}^n, t > 0\right\}, \\
u(x, 0) = f(x) & \text{a.e. on } \mathbb{R}^n.
\end{cases}$$
\end{example}
\chapter{Fourier Transform}
\section{The Fourier Transform on $L^1$ space}
\begin{definition}
    Let $f$ be a function in $L^{1}\left(\mathbb{R}^n\right)$. The Fourier transform of $f$, denoted by $\hat{f}(\xi)$, is defined as

$$\hat{f}(\xi)=\int_{\mathbb{R}^n} f(x) e^{-2 \pi i x \cdot \xi} \, dx, \quad \text{for } \xi \in \mathbb{R}^n.$$
\end{definition}
\begin{proposition}
    Suppose $f \in L^{1}\left(\mathbb{R}^n\right)$. Then the following properties hold:

(a) The $L^{\infty}$-norm of $\hat{f}$ is bounded by the $L^{1}$-norm of $f$, i.e., $\|\hat{f}\|_{\infty} \leqslant\|f\|_1$.

(b) The function $\hat{f}$ is uniformly continuous on $\mathbb{R}^n$.

(c) As $|\xi|$ approaches infinity, $\hat{f}(\xi)$ tends to zero. This is known as the Riemann-Lebesgue Lemma.

(d) If $f$ and $g$ are functions in $L^{1}\left(\mathbb{R}^n\right)$ such that their product $f \cdot g$ is also in $L^{1}\left(\mathbb{R}^n\right)$, then the Fourier transform of their sum is the product of their Fourier transforms, i.e., $\widehat{f+g}=\hat{f} \cdot \hat{g}$.
\end{proposition}
\begin{proof}
    (c):
Consider the following manipulation of the Fourier transform:

$$\begin{aligned}
\hat{f}(\xi) &= \int_{\mathbb{R}^n} f(x) e^{-2 \pi i x \cdot \xi} \, dx \\
&= \int_{\mathbb{R}^n} f(x) e^{-2 \pi i x \cdot \xi} \cdot (-1) e^{-2 \pi i \xi \cdot \frac{\xi}{2|\xi|^2}} \, dx \\
&= -\int_{\mathbb{R}^n} f(x) e^{-2 \pi i \xi \cdot \left(x + \frac{\xi}{2|\xi|^2}\right)} \, dx \\
&= -\int_{\mathbb{R}^n} f\left(x - \frac{\xi}{2|\xi|^2}\right) e^{-2 \pi i x \cdot \xi} \, dx.
\end{aligned}$$
Then, we have
$$\begin{aligned}
|\hat{f}(\xi)| &= \frac{1}{2}|\hat{f}(\xi) + \hat{f}(\xi)| \\
&= \frac{1}{2}\left|\int_{\mathbb{R}^n} \left[f(x) - f\left(x - \frac{\xi}{2|\xi|^2}\right)\right] e^{-2 \pi i x \cdot \xi} \, dx\right| \\
&\leqslant \frac{1}{2} \int_{\mathbb{R}^n} \left|f(x) - f\left(x - \frac{\xi}{2|\xi|^2}\right)\right| \, dx \\
&\rightarrow 0 \quad \text{as } |\xi| \rightarrow \infty.
\end{aligned}$$

\end{proof}
\begin{proposition}
    Let $f$ be a function in $L^{1}(\mathbb{R})$. Then the following properties of the Fourier transform hold:

(1) The Fourier transform of $f(x-b)$ is $e^{-2 \pi i \xi \cdot b} \hat{f}(\xi)$.

(2) The Fourier transform of $e^{2 \pi i x \cdot h} f(x)$ is $\hat{f}(\xi-h)$.

(3) For any positive real number $t$, the Fourier transform of $t^{-n} f(t^{-1} x)$ is $\hat{f}(t \xi)$.

(4) Let $\rho$ be an orthogonal transformation on $\mathbb{R}^n$, i.e., a linear transformation that preserves the inner product, satisfying $\rho(x) \cdot \rho(y) = x \cdot y$. Then the Fourier transform of $f \circ \rho$ is $\hat{f} \circ \rho(\xi)$.

(5) If $f$ is a radial function, then $\hat{f}$ is also radial.
\end{proposition}
\begin{proof}
(4):

    $$\begin{aligned}
(f \circ \rho)^{\wedge}(\xi) &= \int_{\mathbb{R}^n} f(\rho x) e^{-2 \pi i x \cdot \xi} \, dx \\
&\overset{y = \rho x}{=} \int_{\mathbb{R}^n} f(y) e^{-2 \pi i \rho^{-1} y \cdot \xi} \, dy \\
&= \int_{\mathbb{R}^n} f(y) e^{-2 \pi i y \cdot \rho \xi} \, dy \\
&= \hat{f}(\rho \xi).
\end{aligned}$$

(5) To show that $\hat{f}(\xi_1) = \hat{f}(\xi_2)$ when $|\xi_1| = |\xi_2|$, we can use a rotation $\rho$ such that $\rho \xi_1 = \xi_2$. Then, by property (4), we have

$$\hat{f}(\xi_2) = \hat{f}(\rho \xi_1) = (f \circ \rho)^{\wedge}(\xi_1) = \hat{f}(\xi_1).$$
\end{proof}
\begin{theorem}
    Let $f \in L^{1}(\mathbb{R})$. Then,

(1) $\frac{\partial \hat{f}(\xi)}{\partial \xi_k}=\left(-2 \pi i x_k f(x)\right)^{\wedge}(\xi)$ provided that $x_k f \in L^1$.

(2) If $f \in C^{1} \cap C_0$ and $\frac{\partial f}{\partial x_k} \in L^1$, then $\left(\frac{\partial f}{\partial x_k}\right)^{\wedge}(\xi)=2 \pi i \xi_k \hat{f}(\xi)$.

\noindent\textbf{Note}: Here, $C_0$ denotes the set of continuous functions that vanish at infinity, i.e., $C_0=\{f \in C(x): \forall \varepsilon>0, \{x | |f(x)| \geqslant \varepsilon\} \text{ is compact }\}$.
\end{theorem}
\begin{proof}
    (1): Consider $h=(0, \ldots, 0, h_k, 0, \ldots, 0)$ where $h_k$ is the $k^{\text{th}}$ element. We have

$$
\frac{\hat{f}(\xi+h)-\hat{f}(\xi)}{h_k}=\int_{\mathbb{R}^n}\frac{e^{-2 \pi i x_k h_k}-1}{h_k} f(x) e^{-2 \pi i x \cdot \xi} \, dx.
$$
Observe that $\left|\frac{e^{-2 \pi i x_k h_k}-1}{h_k} f(x) e^{-2 \pi i x \cdot \xi}\right| \leqslant 2 \pi |x_k f(x)|$.
By the Dominated Convergence Theorem (D.C.T.),
$$
\lim_{{h_k \to 0}} \frac{\hat{f}(\xi+h)-\hat{f}(\xi)}{h_k}=\int_{\mathbb{R}^n} -2 \pi i x_k f(x) e^{-2 \pi i x \cdot \xi} \, dx = \left(-2 \pi i x_k f(x)\right)^{\wedge}(\xi).
$$
\end{proof}
\begin{corollary}
    For $\alpha \in \mathbb{Z}_+^n$ and $D^\alpha = \left(\partial_{x_1}\right)^{\alpha_1} \cdots \left(\partial_{x_n}\right)^{\alpha_n}$, let $P(x) = \sum_{|\alpha| \leq d} a_\alpha x^\alpha$ where $|\alpha| = \alpha_1 + \cdots + \alpha_n$ and $x^\alpha = x_1^{\alpha_1} x_2^{\alpha_2} \ldots x_n^{\alpha_n}$.
Define the differential operator $P(D) = \sum_{|\alpha| \leqslant d} a_\alpha D^\alpha$. Then,
$$
P(D) \hat{f}(\xi) = (P(-2 \pi i x) f(x))^{\wedge}(\xi).
$$
Moreover, for $f \in S(\mathbb{R}^n)$ (the Schwartz space),
$$
(P(D) f)^{\wedge}(\xi) = P(2 \pi i \xi) \hat{f}(\xi).
$$
\end{corollary}
\begin{definition}[Inverse Fourier Transform]
    If $f \in L^{1}$, the inverse Fourier transform of $f$ is defined as

$$
\check{f}(x) = \int_{\mathbb{R}^n} f(\xi) e^{2 \pi i \xi \cdot x} \, d\xi \, (= \hat{f}(-x)).
$$
\end{definition}
\begin{lemma}
    If $f, g \in L^{\prime}$, then $\int \hat{f}(\xi) g(\xi) d\xi = \int f(x) \hat{g}(x) dx$ (the multiplication formula).
\end{lemma}
\begin{proof}
    $$\begin{aligned}
\int_{\mathbb{R}^n} \hat{f}(\xi) g(\xi) d\xi &= \iint_{{\mathbb{R}^n}\times{\mathbb{R}^n}} f(x) e^{-2 \pi i x \cdot \xi} dx g(\xi) d\xi \\
&= \iint_{{\mathbb{R}^n}\times{\mathbb{R}^n}} g(\xi) e^{-2 \pi i x \cdot \xi} d\xi f(x) dx \\
&= \int_{\mathbb{R}^n} f(x) \hat{g}(x) dx.
\end{aligned}
$$
\end{proof}
\begin{lemma}
 $\left(e^{-\pi|x|^2}\right)^{\wedge}(\xi) = e^{-\pi|\xi|^2}$.
\end{lemma}
\begin{proof}
$$
    \begin{aligned}
\left(e^{-\pi|x|^2}\right)^{\wedge}(\xi) &= \int_{\mathbb{R}^n} e^{-\pi|x|^2} e^{-2 \pi i x \cdot \xi} dx \\
&= \prod_{i=1}^n \int_{\mathbb{R}} e^{-\pi x_i^2} e^{-2 \pi i x_i \xi_i} dx_i.
\end{aligned}
$$
It suffices to show $\left(e^{-\pi x^2}\right)^{\wedge}(\xi) = e^{-\pi \xi^2}$ for $x, \xi \in \mathbb{R}^1$.

The function $f(x) = e^{-\pi x^2}$ is the solution of the initial value problem (I.V.P):
$$
\begin{cases}
u' + 2\pi x u = 0, \\
u(0) = 1.
\end{cases}
$$
If $f \in S$ satisfies the initial value problem, then $\hat{f}$ also satisfies the same initial value problem. Indeed,
$$
\begin{aligned}
\hat{f}(0) &= \int_{\mathbb{R}^n} f(x) dx = 1, \\
0 &= \left(f' + 2\pi x f\right)^{\wedge}(\xi) \\
&= 2\pi i \xi \hat{f}(\xi) + \frac{1}{-i}(\hat{f})'(\xi) \\
&= i(\hat{f}(\xi) + 2\pi \xi \hat{f}(\xi)).
\end{aligned}
$$
Therefore, by uniqueness, $\hat{f} = f$.

\noindent\textbf{Remark}: $\left(e^{-\pi a|x|^2}\right)^{\wedge}(\xi) = a^{-\frac{n}{2}} e^{-\pi \frac{|\xi|^2}{a}}$ for $a > 0$.
\end{proof}
\begin{theorem}[The Fourier inversion theorem for $L^{1}$ functions]\label{thm2.1}
    If $f, \hat{f} \in L^1\left(\mathbb{R}^n\right)$, then $(\hat{f})^\vee = f$ almost everywhere (a.e.).

\end{theorem}
\noindent\textbf{Remark}: Since $\hat{f} \in L^{1}$, then $(\hat{f})^\vee \in C_0$. Modify $f$ on a set of measure zero such that
$$
(\hat{f})^\vee(x) = f(x) \quad \forall x.
$$
\begin{definition}
    $G_{\varepsilon}(f) = \int_{\mathbb{R}^n} f(x) e^{-\varepsilon|x|^2} dx$ is the Gauss means of $\int_{\mathbb{R}^n} f(x) dx$.
\end{definition}
\begin{theorem}
    If $f \in L^{1}\left(\mathbb{R}^n\right)$, then
$$
\left\| \int_{\mathbb{R}^n} \hat{f}(\xi) e^{2 \pi i x \cdot \xi} e^{-4 \pi^2 \varepsilon^2|\xi|^2} d\xi - f(x) \right\|_{L^{1}(dx)} \rightarrow 0
$$
as $\varepsilon \rightarrow 0$.
\end{theorem}
\begin{proof}
    $$
\begin{aligned}
\int_{\mathbb{R}^n}\hat{f}(\xi) e^{2 \pi i x \xi} e^{-4 \pi^2 \varepsilon^2|\xi|^2} d\xi &= \int f(y) \left(e^{2 \pi i x \cdot \xi} e^{-4 \pi^2 \varepsilon^2|\xi|^2}\right)^{\wedge}(y) dy \\
&= \int_{\mathbb{R}^n} f(y) \left(e^{-4 \pi^2 \varepsilon^2|\xi|^2}\right)^{\wedge}(y-x) dy \\
&= \int_{\mathbb{R}^n} f(y) \varepsilon^{-n}(4 \pi)^{-\frac{n}{2}} e^{-\frac{1}{4}\left|\frac{y-x}{\varepsilon}\right|^2} dy \\
&= \int_{\mathbb{R}^n} f(y) \varphi_{\varepsilon}(x-y) dy \\
&= f * \varphi_{\varepsilon}(x),
\end{aligned}
$$
where $\varphi(x) = (4 \pi)^{-\frac{n}{2}} e^{-\frac{1}{4}|x|^2} \in L^1$ and $\int_{\mathbb{R}^n}\varphi(x) dx= 1$.
\end{proof}
Next we prove Theorem \ref{thm2.1}
\begin{proof}
    By the previous theorem, there exists a sequence $\left\{{\varepsilon_k}\right\}_{k=1}^{\infty}$ converging to 0 such that

$$\lim_{{\varepsilon_k \rightarrow 0}} \int \hat{f}(\xi) e^{2 \pi i x \cdot \xi} e^{-4 \pi^2 \varepsilon_k^2|\xi|^2} \, d\xi = f(x) \text{ for almost every } x.$$

Since $\hat{f} \in L^{\prime}$, by the Dominated Convergence Theorem, we have

$$f(x) = \int \lim_{{\varepsilon_k \rightarrow 0}} \hat{f}(\xi) e^{2 \pi i x \cdot \xi} e^{-4 \pi^2 \varepsilon_k^2|\xi|^2} \, d\xi = \int \hat{f}(\xi) e^{2 \pi i x \cdot \xi} \, d\xi = (\hat{f})^{\vee}(x).$$
\end{proof}
\begin{corollary}
    If $f_1, f_2 \in L^{\prime}$ and $\hat{f}_1(\xi) = \hat{f}_2(\xi)$, then $f_1(x) = f_2(x)$ for almost every $x$.

\end{corollary}
\begin{proof}
    Set $f = f_1 - f_2$. Then $f \in L^1$ and $\hat{f} = \widehat{f_1 - f_2} = \hat{f_1} - \hat{f_2} = 0 \in L^1$.
By Fourier inversion, $f = (\hat{f})^{\vee} = 0$, thus $f_1 = f_2$ almost everywhere.
\end{proof}
\noindent\textbf{Notation}: Let $\mathcal{F}f(\xi) = \hat{f}(\xi)$ denote the Fourier transform of $f$ evaluated at $\xi$.
\section{The Fourier Transform on $L^2$ and $L^p,(1<p<2)$ space}
\begin{theorem}
    Suppose that $|\varphi(x)| \leq \frac{c}{(1+|x|)^{n+\varepsilon_0}}$ for some constants $c, \varepsilon_0 > 0$ and $\int_{\mathbb{R}^n} \varphi(x) \, dx = a$. If $f \in L^p$ for $1 \leq p \leq \infty$, then
$$\lim_{\varepsilon \rightarrow 0} f * \varphi_{\varepsilon}(x) = a f(x)$$
holds for every $x$ in the Lebesgue set of $f$.

\noindent\textbf{Remark}: The Lebesgue set $L_f$ of $f$ is defined as the set of points $x$ where $f(x)$ is finite and
$$\lim_{r \rightarrow 0} \frac{1}{r^n} \int_{|y| < r} |f(x-y) - f(x)| \, dy = 0.$$
\end{theorem}
\begin{proof}
    Let $x \in L_f$. For any $\delta > 0$, there exists $\eta > 0$ such that

$$\frac{1}{r^n} \int_{|y| < r} |f(x-y) - f(x)| \, dy \leq \delta$$
for all $r \leq \eta$. Now consider the expression
$$\left|f * \varphi_{\varepsilon}(x) - a f(x)\right| = \left|\int_{\mathbb{R}^n} f(x-y) \varphi_{\varepsilon}(y) \, dy - \int_{\mathbb{R}^n} f(x) \varphi_{\varepsilon}(y) \, dy\right|.$$
This can be rewritten as
$$\left|\int_{\mathbb{R}^n} (f(x-y) - f(x)) \varphi_{\varepsilon}(y) \, dy\right|.$$
We split the integral into two parts:
$$\underbrace{\int_{|y| \leq \eta} |f(x-y) - f(x)| |\varphi_{\varepsilon}(y)| \, dy}_{I_1} + \underbrace{\int_{|y| > \eta} |f(x-y) - f(x)| |\varphi_{\varepsilon}(y)| \, dy}_{I_2}.$$
We claim that $I_1 \leq A \delta$, where $A$ is independent of $\varepsilon$, and $I_2 \rightarrow 0$ as $\varepsilon \rightarrow 0$. Since $\left|f * \varphi_{\varepsilon}(x)-a f(x)\right| \leqslant I_1+I_2 \leqslant A \delta+I_2$, we have
$$
\limsup _{\varepsilon \rightarrow 0}\left|f * \varphi_{\varepsilon}(x)-a f(x)\right| \leq A \delta.
$$
As $\delta$ is arbitrary, we conclude that $\lim _{\varepsilon \rightarrow 0} f * \varphi_{\varepsilon}(x)=a f(x)$.
To estimate $I_1$, let $K \in \mathbb{N}$ be fixed such that $2^K \leqslant \eta / \varepsilon < 2^{K+1}$ when $\eta / \varepsilon \geqslant 2$. We define the set $B(0, \eta)$ as follows:
$$
B(0, \eta) = 
\begin{cases}
B(0, 2^{-k} \eta) \cup \left(\bigcup_{i=1}^K \left\{ y \mid 2^{-i} \eta \leq |y| < 2 \cdot 2^{-i} \eta \right\}\right), & \text{if } \eta / \varepsilon \geqslant 2, \\
B(0, \eta), & \text{if } \eta / \varepsilon < 2.
\end{cases}
$$
Case 1: $\eta / \varepsilon < 2$. In this case, we have
$$
I_1 \leqslant c \varepsilon^{-n} \int_{B(0, \eta)}|f(x-y)-f(x)| \, dy \leqslant c \varepsilon^{-n} \delta \eta^n \leqslant c \delta.
$$
Case 2: $\eta / \varepsilon \geqslant 2$. On the $k$-th annulus, we estimate
$$
\left|\varphi_{\varepsilon}(y)\right| = \varepsilon^{-n}\left|\varphi\left(\varepsilon^{-1} y\right)\right| \leqslant C \varepsilon^{-n} \frac{1}{\left|\varepsilon^{-1} y\right|^{n+\varepsilon_0}} \leqslant C \varepsilon^{\varepsilon_0} \frac{1}{\left(2^k \eta\right)^{n+\varepsilon_0}}.
$$
On the ball $B\left(0, 2^{-k} \eta\right)$, we use the estimate $\left|\varphi_{\varepsilon}(y)\right| \leqslant C \varepsilon^{-n}$.
Thus,
$$
\begin{aligned}
I_1 &\leqslant \sum_{k=1}^K c \varepsilon^{\varepsilon_0} \frac{1}{\left(2^k \eta\right)^{n+\varepsilon_0}} \delta \left(2 \cdot 2^{-k} \eta\right)^n + c \varepsilon^{-n} \delta \left(2^{-K} \eta\right)^n \\
&=c \delta \frac{\varepsilon^{\varepsilon_0}}{\eta^{\varepsilon_0}} \sum_{k=1}^K 2^{-k(n+\varepsilon_0-n)} + c \delta \left(2^{-K} \frac{\eta}{\varepsilon}\right)^n \\
&= c \delta \frac{\varepsilon^{\varepsilon_0}}{\eta^{\varepsilon_0}} \sum_{k=1}^K 2^{-k\varepsilon_0} + c \delta \left(2^{-K} \frac{\eta}{\varepsilon}\right)^n \\
&\leq c \delta \frac{\varepsilon^{\varepsilon_0}}{\eta^{\varepsilon_0}} \frac{1 - 2^{-K\varepsilon_0}}{1 - 2^{-\varepsilon_0}} + c \delta \\
&\leq c \delta \frac{1}{1 - 2^{-\varepsilon_0}} + c \delta = A \delta
\end{aligned}
$$
As for $I_2$, if $p'$ is the conjugate exponent to $p$ and $x = \textbf{1}_{\{|y|: |y| > \eta\}}$, by Hölder's inequality we have:
$$
\begin{aligned}
I_2 &\leqslant \int (|f(x-y)| + |f(x)|) \left| \textbf{1}_{\{|y|: |y| > \eta\}} \varphi_{\varepsilon}(y) \right| dy \\
&\leqslant \|f\|_p \left\| \textbf{1}_{\{|y|: |y| > \eta\}} \varphi_{\varepsilon} \right\|_{p'} + |f(x)| \left\| \textbf{1}_{\{|y|: |y| > \eta\}} \varphi_{\varepsilon} \right\|_1
\end{aligned}
$$
It suffices to show $\left\| \textbf{1}_{\{|y| > \eta\}} \varphi_{\varepsilon} \right\|_q \rightarrow 0$ as $\varepsilon \rightarrow 0$.
If $q = \infty$, then
$$
\left\| 1_{\{|y| > \eta\}} \varphi_{\varepsilon} \right\|_{\infty} \leqslant \varepsilon^{-n} \frac{c}{(1 + \varepsilon^{-1} \eta)^{n+\varepsilon_0}} \leq C \eta^{-n-\varepsilon_0} \varepsilon^{\varepsilon_0} \rightarrow 0
$$
as $\varepsilon \rightarrow 0$.

If $q < \infty$, then
$$
\begin{aligned}
\left\| 1_{\{|y| > \eta\}} \varphi_{\varepsilon} \right\|_q^q &= \int_{|y| > \eta} \left| \varepsilon^{-n} \varphi\left(\varepsilon^{-1} y\right) \right|^q dy \\
&= \varepsilon^{-nq} \int_{|z| \geqslant \frac{\eta}{\varepsilon}} |\varphi(z)|^q \cdot \varepsilon^n dz \\
&\leq C \varepsilon^{n(1-q)} \int_{|z| \geqslant \frac{\eta}{\varepsilon}} \frac{1}{(1 + |z|)^{(n+\varepsilon_0)q}} dz \\
&\leq C \varepsilon^{n(1-q)} \int_{r = \frac{\eta}{\varepsilon}}^{\infty} \frac{r^{n-1}}{(1 + r)^{(n+\varepsilon_0)q}} dr \\
&\leq C \eta^{n - (n + \varepsilon_0)q} \varepsilon^{\varepsilon_0 q} \longrightarrow 0
\end{aligned}
$$
as $\varepsilon \rightarrow 0$.
\end{proof} 
\begin{corollary}
    Suppose $f \in L^1\left(\mathbb{R}^n\right)$ and $\hat{f} \geq 0$. If $f$ is continuous at 0, then $\hat{f} \in L^1$ and $f(x)=\int \hat{f}(\xi) e^{2 \pi i x \cdot \xi} d \xi$ almost everywhere. In particular, $f(0)=\int \hat{f}(\xi) d \xi$.
\end{corollary}
\begin{proof}
     Since $f$ is continuous at 0, we have that $0 \in L_f$. Recall that
$$
\int \hat{f}(\xi) e^{2 \pi i x \cdot \xi} e^{-4 \pi^2 \varepsilon^2 |\xi|^2} d \xi=\int f * \varphi_{\varepsilon}(x) dx
$$
with $\varphi(x)=(4 \pi)^{-\frac{n}{2}} e^{-\frac{|x|^2}{4}}$ and $\varphi_{\varepsilon}(x) = \varepsilon^{-n} \varphi(x/\varepsilon)$. By a previous result , we have at $x=0$
$$
f(0)=\lim _{\varepsilon \rightarrow 0} \int f * \varphi_{\varepsilon}(0) dx = \lim _{\varepsilon \rightarrow 0} \int \hat{f}(\xi) e^{-4 \pi^2 \varepsilon^2|\xi|^2} d \xi
$$
Hence,
$$
\|\hat{f}\|_1=\int \hat{f}(\xi) d\xi = \int \lim_{\varepsilon \rightarrow 0} \hat{f}(\xi) e^{-4 \pi^2 \varepsilon^2|\xi|^2} d \xi \leq \liminf _{\varepsilon \rightarrow 0} \int \hat{f}(\xi) e^{-4 \pi^2 \varepsilon^2|\xi|^2} d \xi=f(0)<\infty
$$
thus $\hat{f} \in L^1$. By the Dominated Convergence Theorem (D.C.T.), 
$$
f(0)=\int \lim _{\varepsilon \rightarrow 0} \hat{f}(\xi) e^{-4 \pi^2 \varepsilon^2|\xi|^2} d \xi=\int \hat{f}(\xi) d \xi
$$
\end{proof}
Now, let's proceed to define the Fourier Transform on $L^2$.
\begin{theorem}
    If $f \in L^1 \cap L^2$, then $\|\hat{f}\|_2=\|f\|_2<\infty$ ($\hat{f} \in L^2$).
\end{theorem}
\begin{proof}
    Let $g(x)=\overline{f(-x)}$. Then $h=f * g \in L^1$ ($\|h\|_1=\|f * g\|_1 \leq\|f\|_1\|g\|_1=\|f\|_1^2$).
$h$ is bounded ($\|h\|_{\infty} \leq\|f\|_2\|g\|_2 = \|f\|_2^2$) and uniformly continuous.
Thus $\hat{h}=\hat{f} * \hat{g}=\hat{f} \hat{g}=\hat{f} \overline{\hat{f}}=|\hat{f}|^2 \geq 0$.
By the previous corollary, $\hat{h} \in L^1$ and $h(0)=\int \hat{h}(\xi) d \xi$.
We thus have $\int|\hat{f}|^2 d\xi = h(0) = f * g(0) = \int f(x) g(0-x) dx = \int f(x) \overline{f(x)} dx = \int|f|^2 dx$.

\end{proof}
\begin{theorem}
    Let $f \in L^2(\mathbb{R}^n)$ and $\left\{g_n\right\} \subset L^1(\mathbb{R}^n) \cap L^2(\mathbb{R}^n)$ with $g_n \rightarrow f$ in $L^2(\mathbb{R}^n)$. Then $\hat{g}_n$ converges to a function $\mathcal{F}f$ in $L^2(\mathbb{R}^n)$. $\mathcal{F}f$ is independent of the particular sequence $\{g_n\}$ and $\mathcal{F}f$ is called the $L^2$ Fourier transform of $f$ on $L^2(\mathbb{R}^n)$.
The Fourier transform on $L^2(\mathbb{R}^n)$ will be denoted by $\mathcal{F}$, and we shall use the notation $\hat{f}=\mathcal{F}f$ whenever $f \in L^2(\mathbb{R}^n)$.
\end{theorem}
\begin{proof}
We have
$\left\|\hat{g}_n-\hat{g}_m\right\|_2 = \left\|(g_n-g_m)^\wedge\right\|_2 = \left\|g_n-g_m\right\|_2 \rightarrow 0$as $m, n \rightarrow \infty$.
Thus $\{\hat{g}_n\}$ is an Cauchy sequence in $L^2(\mathbb{R}^n)$. Then there exists an $L^2(\mathbb{R}^n)$ function, denoted by $\mathcal{F}f$, such that $\hat{g}_n \xrightarrow{L^2} \mathcal{F}f$.
Assume $\{g_n\}$ and $\{\widetilde{g}_n\}$ both converge to $f$ in $L^1(\mathbb{R}^n) \cap L^2(\mathbb{R}^n)$ with respect to the $L^2$-norm.
Consider the sequence $\{g_1, \tilde{g}_1, g_2, \tilde{g}_2, \ldots, g_n, \tilde{g}_n, \ldots\}$ in $L^1(\mathbb{R}^n) \cap L^2(\mathbb{R}^n)$ converging to $f$ in $L^2(\mathbb{R}^n)$.
Hence, there exists $h \in L^2(\mathbb{R}^n)$ such that $\{\hat{g}_1, \hat{\tilde{g}}_1, \ldots, \hat{g}_n, \widehat{\tilde{g}}_n, \ldots\} \xrightarrow{L^2} h$.
Both $\hat{g}_n \xrightarrow{L^2} h$ and $\widehat{\tilde{g}}_n \xrightarrow{L^2} h$ converge to the same function in $L^2(\mathbb{R}^n)$. 
\end{proof}
\begin{theorem}[The Plancherel Theorem]
   For any function $f$ in the space $L^2(\mathbb{R}^n)$, the Fourier transform $\mathcal{F}f$ satisfies the equality $\|\mathcal{F}f\|_2 = \|f\|_2$.
\end{theorem}
\begin{proof}
Assume we have a sequence of functions $g_n$ that converge to $f$ in the $L^2$ norm, and each $g_n$ belongs to the intersection of $L^1(\mathbb{R}^n)$ and $L^2(\mathbb{R}^n)$. Then, the Fourier transforms $\hat{g}_n$ converge to $\mathcal{F}f$ in the $L^2$ norm.\\
Since the Fourier transform preserves the $L^2$ norm for each $g_n$, i.e., $\|\hat{g}_n\|_2 = \|g_n\|_2$, we can take the limit as $n$ approaches infinity to obtain $\|\mathcal{F}f\|_2 = \|f\|_2$.
\end{proof}
\begin{corollary}
    The Fourier transform $\mathcal{F}$ is injective.
\end{corollary}
Then we will show $\mathcal{F}$ be a linear operator. For all $f, g \in L^2(\mathbb{R}^n)$, there exist sequences ${f_n}$ and ${g_n}$ such that $f_n$ converges to $f$ in $L^2$, $g_n$ converges to $g$ in $L^2$, and $f_n \cdot g_n$ belongs to both $L^1(\mathbb{R}^n)$ and $L^2(\mathbb{R}^n)$. We have that:
$$\mathcal{F}f = \lim_{{n \to \infty} } \hat{f}_n \quad \mathcal{F}g = \lim_{{n \to \infty} } \hat{g}_n$$
Consequently, the linearity of $\mathcal{F}$ is established as follows:
$$\begin{aligned}
    \mathcal{F}(f+g) &= \lim_{{n \to \infty} \atop {\text{in } L^2}} \widehat{f_n + g_n}
&= \lim_{{n \to \infty} \atop {\text{in } L^2}} (\hat{f}_n + \hat{g}_n)
&= \lim_{{n \to \infty} \atop {\text{in } L^2}} \hat{f}_n + \lim_{{n \to \infty} \atop {\text{in } L^2}} \hat{g}_n
&= \mathcal{F}f + \mathcal{F}g
\end{aligned}$$
Thereby, we conclude that $\mathcal{F}$ is a unique bounded linear operator mapping from $L^2(\mathbb{R}^n)$ into $L^2(\mathbb{R}^n)$. Additionally, for any function $f$ that belongs to both $L^1(\mathbb{R}^n)$ and $L^2(\mathbb{R}^n)$, $\mathcal{F}f$ is equivalent to its Fourier transform $\hat{f}$.

To demonstrate uniqueness, let us assume the existence of two distinct bounded linear operators, $F_1$ and $F_2$, both defined on $L^2(\mathbb{R}^n)$ with values in $L^2(\mathbb{R}^n)$. Furthermore, let these operators satisfy the property that for all functions $f$ in the intersection of $L^1(\mathbb{R}^n)$ and $L^2(\mathbb{R}^n)$, $F_i f = \hat{f}$ for $i = 1,2$.

For any function $g$ in $L^2(\mathbb{R}^n)$, we can construct a sequence ${g_n}$ belonging to the intersection of $L^1(\mathbb{R}^n)$ and $L^2(\mathbb{R}^n)$ such that $g_n$ approaches $g$ in the $L^2$ norm. Now, considering the differences between $F_1 g$ and $F_2 g$ in the $L^2$ norm, we have:
$$\|F_1 g - F_2 g\|_2 \leqslant \|F_1 g - F_1 g_n\|_2 + \|F_1 g_n - F_2 g_n\|_2 + \|F_2 g_n - F_2 g\|_2$$
The middle term, $\|F_1 g_n - F_2 g_n\|_2$, vanishes since both operators agree on $g_n$ (by our assumption). Thus,
$$\|F_1 g - F_2 g\|_2 \leqslant c\|g - g_n\|_2 + c'\|g_n - g\|_2$$
for some constants $c$ and $c'$ depending on the boundedness of the operators. As $n$ approaches infinity, this expression tends to zero, implying the equality of $F_1$ and $F_2$. This establishes the uniqueness of the operator $\mathcal{F}$.
\begin{proposition}
    The Fourier transform $\mathcal{F}: L^2 \rightarrow L^2$ is surjective.
\end{proposition}
\begin{proof}
    Since $\mathcal{F}$ is an isometry, its range $R(\mathcal{F})$ constitutes a closed subspace of $L^2$ (as established in Claim 1). Furthermore, the Schwartz space $S$ is contained within $R(\mathcal{F})$ (as shown in Claim 2). Therefore, the image of $\mathcal{F}$ is indeed the entirety of $L^2$.

\noindent\textbf{Claim 1}:\\
Consider a sequence $\{\mathcal{F} f_n\}$ in $R(\mathcal{F})$ where $f_n \in L^2$ and $\mathcal{F} f_n$ converges to $g$ in the $L^2$ norm. We can deduce that $\{\mathcal{F} f_n\}$ is a Cauchy sequence in $L^2$. Consequently, $\{f_n\}$ is also a Cauchy sequence in $L^2$ and converges to some $f \in L^2$. By continuity of $\mathcal{F}$, we have $\mathcal{F} f_n \rightarrow \mathcal{F} f$ in $L^2$ and thus $g = \mathcal{F} f$. This establishes that $g$ belongs to $R(\mathcal{F})$.

\noindent\textbf{Claim 2}:\\
It is a known fact that the \textbf{Fourier transform maps the Schwartz space $S$ onto itself bijectively.}
To elaborate, if $f$ belongs to $S$, then it is both integrable and bounded. For any multi-indices $\alpha$ and $\beta$, the function $\widehat{D^\alpha(x^\beta f)}$ is bounded since $D^\alpha(x^\beta f)$ also belongs to $S$. We have the identity $\widehat{D^\alpha(x^\beta f)} = C_{\alpha, \beta} \xi^\alpha D^\beta \hat{f}$ which implies that $\xi^\alpha D^\beta \hat{f}$ is bounded. This, in turn, means that $\hat{f}$ belongs to $S$.

The injectivity of $\mathcal{F}$ on $S$ follows from the fact that if $f_1, f_2 \in S$ and $\mathcal{F} f_1 = \mathcal{F} f_2$, then $f_1$ and $f_2$ must be equal almost everywhere. Hence, they are equivalent.

To show the surjectivity of $\mathcal{F}$ on $S$, consider any $f \in S$. Define $F(x) = f(-x)$ and let $g = \hat{F}$. It can be shown that $g$ belongs to $S$ and that $\hat{g}(\xi) = f(\xi)$. This demonstrates that $\mathcal{F}$ is surjective onto $S$.
\end{proof}
\begin{theorem}
    The Fourier transform is a unitary operator on $L^2$. \\
    \noindent\textbf{Unitary}: A linear operator on $L^2$ that is an isometry and maps onto $L^2$.
\end{theorem}
\begin{corollary}
    The Fourier transform on $L^2$ preserves inner products: $\langle\mathcal{F}f, \mathcal{F}g\rangle = \langle f, g\rangle$ for all $f, g \in L^2$.
\end{corollary}
\begin{proof}
    By the polarization identity, we have
$$
\langle f, g\rangle = \int_{\mathbb{R}^n} f \bar{g} = \frac{1}{4}\left(\|f+g\|_2^2 - \|f-g\|_2^2 + i\|f+ig\|_2^2 - i\|f-ig\|_2^2\right).
$$
Since $\mathcal{F}$ is an isometry, it follows that $\|\mathcal{F}f\|_2 = \|f\|_2$ for all $f \in L^2$. Therefore, applying the polarization identity to $\mathcal{F}f$ and $\mathcal{F}g$, we obtain
$$
\langle \mathcal{F}f, \mathcal{F}g\rangle = \frac{1}{4}\left(\|\mathcal{F}f+\mathcal{F}g\|_2^2 - \|\mathcal{F}f-\mathcal{F}g\|_2^2 + i\|\mathcal{F}f+i\mathcal{F}g\|_2^2 - i\|\mathcal{F}f-i\mathcal{F}g\|_2^2\right) = \langle f, g\rangle.
$$
This completes the proof that the Fourier transform preserves inner products on $L^2$.
\end{proof}
\begin{theorem}
    The inverse of the Fourier transform, denoted by $\mathcal{F}^{-1}$, can be obtained by letting $(\mathcal{F}^{-1}g)(x) = (\mathcal{F}g)(-x)$ for all $g \in L^2(\mathbb{R}^n)$.
\end{theorem}
\begin{proof}
    Suppose first that $g \in S(\mathbb{R}^n)$, the Schwartz space of rapidly decreasing functions. Then there exists $f \in S(\mathbb{R}^n)$ such that $g = \hat{f}$, where $\hat{f}$ denotes the Fourier transform of $f$. By the Fourier inversion formula for functions in the Schwartz space, we have
$$
f = \mathcal{F}^{-1}g = (\hat{f})^\vee = \hat{g}(-x) = (\mathcal{F}g)(-x).
$$
This shows that $(\mathcal{F}^{-1}g)(x) = (\mathcal{F}g)(-x)$ holds for all $g \in S(\mathbb{R}^n)$.

Now let $g \in L^2(\mathbb{R}^n)$ be arbitrary. By density of the Schwartz space in $L^2(\mathbb{R}^n)$, there exists a sequence $\{g_k\}$ in $S(\mathbb{R}^n)$ such that $g_k \to g$ in $L^2$ as $k \to \infty$. Using the triangle inequality and the fact that $\mathcal{F}$ is an isometry on $L^2$, we have
$$
\begin{aligned}
&\|(\mathcal{F}g)(-x) - \mathcal{F}^{-1}g(x)\|_{L^2} \\
&\leq \|(\mathcal{F}g)(-x) - (\mathcal{F}g_k)(-x)\|_2 + \|(\mathcal{F}g_k)(-x) - \mathcal{F}^{-1}g_k(x)\|_2 + \|\mathcal{F}^{-1}g_k(x) - \mathcal{F}^{-1}g(x)\|_2 \\
&= \|g - g_k\|_2 + 0 + \|g_k - g\|_2 \to 0 \text{ as } k \to \infty.
\end{aligned}
$$
This shows that $(\mathcal{F}g)(-x) = \mathcal{F}^{-1}g(x)$ almost everywhere, completing the proof.
\end{proof}
If a function $f$ can be expressed as the sum of two functions $f_1$ and $f_2$ where $f_1$ belongs to $L^1$ and $f_2$ belongs to $L^2$, then we write $f = f_1 + f_2$. In this case, we define the Fourier transform of $f$ as $\hat{f} = \hat{f}_1 + \hat{f}_2$.

\textbf{Well-definedness}: Suppose we have another decomposition of $f$ as $g_1 + g_2$ where $g_1 \in L^1$ and $g_2 \in L^2$. Then, it follows that $f_1 + f_2 = g_1 + g_2$. Rearranging this equation, we obtain $f_1 - g_1 = g_2 - f_2$. Since both $L^1$ and $L^2$ are linear spaces, the difference $f_1 - g_1$ belongs to $L^1$ and the difference $g_2 - f_2$ belongs to $L^2$. Therefore, the Fourier transform of $f_1 - g_1$ exists and is equal to the Fourier transform of $g_2 - f_2$. This implies that $\hat{f}_1 + \hat{f}_2 = \hat{g}_1 + \hat{g}_2$, showing that the definition of the Fourier transform for functions in $L^1 + L^2$ is well-defined.

\begin{definition}
    For $1 \leq p \leq 2$, since $L^p \subset L^1 + L^2$, we can apply the above definition to functions in $L^p$.
\end{definition}
\section{The Fourier Transform on $\mathcal{S}^{\prime}$}
At the begining, we define the \textbf{seminorm} $\|f\|_{\alpha, \beta}=\|x^\alpha D^\beta f\|_{\infty}$. Then we define a topology on $\mathcal{S}$ as follows: a sequence $\{f_k\} \subset \mathcal{S}$ converges in $\mathcal{S}$ to $f$ if and only if
$$
\forall \alpha, \beta \in \mathbb{N}_0^n, \quad \lim_{k \rightarrow \infty} \|f_k - f\|_{\alpha, \beta} = 0.
$$
The space of bounded linear functionals on $\mathcal{S}$, denoted by $\mathcal{S}'$, is called the space of tempered distributions. A linear map $T: \mathcal{S} \rightarrow \mathbb{C}$ belongs to $\mathcal{S}'$ if $\lim_{k \rightarrow \infty} T(\phi_k) = 0$ whenever $\lim_{k \rightarrow \infty} \phi_k = 0$ in $\mathcal{S}$.
\begin{theorem}
    The Fourier transform is a continuous map from $\mathcal{S}$ to $\mathcal{S}$.
\end{theorem}
\begin{proof}
    We have
$$
\|\hat{f}\|_{\alpha, \beta} = \|\xi^\alpha D^\beta \hat{f}(\xi)\|_{\infty} \leq C\|\widehat{D^\alpha(x^\beta f)}(\xi)\|_{\infty} \leq C\|D^\alpha(x^\beta f)\|_1.
$$
The $L^1$ norm can be bounded by a finite linear combination of seminorms of $f$, which implies that the Fourier transform is a continuous map.
\end{proof}
\noindent\textbf{Remark}: Using Leibniz's rule, we can write
$$
D^\alpha(x^\beta f) = \sum_{\alpha_1 + \alpha_2 = \alpha} C_{\alpha_1, \alpha_2} D^{\alpha_1}(x^\beta) D^{\alpha_2}f.
$$
Then,
$$
\begin{aligned}
\|D^{\alpha_1}(x^\beta) D^{\alpha_2}f\|_1 &= \|(1 + |x|^2)^{-N}(1 + |x|^2)^N D^{\alpha_1}(x^\beta) D^{\alpha_2}f\|_1 \\
&\leq \|(1 + |x|^2)^N D^{\alpha_1}(x^\beta) D^{\alpha_2}f\|_{\infty} \|(1 + |x|^2)^{-N}\|_1 \\
&< \infty,
\end{aligned}
$$
since the first factor is a finite linear combination of seminorms of $f$.
\begin{definition}
    The Fourier transform (F.T.) of $T \in S'$ is the tempered distribution given by
    $$
\hat{T}(f) = T(\hat{f}) \quad \text{for } f \in S.
$$
\end{definition}
\noindent\textbf{Remark}:  $\hat{T}$ is a tempered distribution since $f_k \rightarrow 0$ in $S$ implies $\hat{f}_k \rightarrow 0$ in $S$ and hence
$$
T(\hat{f}_k) \rightarrow 0 \Rightarrow \hat{T} \in S'.
$$
~\\
\textbf{\large{Examples of tempered distributions}}

(1) Let $f \in L^p$ with $1 \leqslant p \leqslant \infty$. Define
$$
L(\varphi) = L_f(\varphi) = \int_{\mathbb{R}^n} f(x) \varphi(x) \, dx \quad \text{for } \varphi \in S.
$$
$L_f \in S'$ since $\|L_f(\varphi)\| \leq \|f\|_p \|\varphi\|_q \rightarrow 0$ as $\varphi \rightarrow 0$ in $S$ (where $q$ is the conjugate exponent of $p$). Then,
$$
\hat{L}_f(\varphi) = L_f(\hat{\varphi}) = \int_{\mathbb{R}^n} f(x) \hat{\varphi}(x) \, dx.
$$
If $1 \leqslant p \leqslant 2$ and $f \in L^p$, then for $\varphi \in S$,
$$
\hat{L}_f(\varphi) = \int_{\mathbb{R}^n} f(x) \hat{\varphi}(x) \, dx = \int_{\mathbb{R}^n} \hat{f}(x) \varphi(x) \, dx,
$$
where $\hat{f}$ is the Fourier transform of $f$ in the sense of $L^{p'}$ norm (with $p'$ being the conjugate exponent of $p$). A distribution $u \in S'$ \textbf{coincides with} a function $h$ if
$$
u(\varphi) = \int_{\mathbb{R}^n} h(x) \varphi(x) \, dx \quad \text{for all } \varphi \in S.
$$
In this case, $\hat{L}_f$ coincides with $\hat{f}$.

\noindent\textbf{Note}: For any $p > 2$, there exists an $f \in L^p$ whose Fourier transform as a tempered distribution dose not \textbf{coincides with} a function.

(2) If $\mu$ is a finite Borel measure, the linear functional $L = L_\mu$ defined by
$$
L(\varphi) = L_\mu(\varphi) = \int_{\mathbb{R}^n} \varphi(x) \, d\mu(x) \quad \text{for } \varphi \in S
$$
is a tempered distribution. Then,
$$
\begin{aligned}
\hat{L}_\mu(\varphi) &= L_\mu(\hat{\varphi}) = \int_{\mathbb{R}^n} \hat{\varphi}(\xi) \, d\mu(\xi) \\
&= \int_{\mathbb{R}^n} \int_{\mathbb{R}^n} \varphi(x) e^{-2\pi i x \cdot \xi} \, dx \, d\mu(\xi) \\
&= \int_{\mathbb{R}^n} \left( \int_{\mathbb{R}^n} e^{-2\pi i x \cdot \xi} \, d\mu(\xi) \right) \varphi(x) \, dx.
\end{aligned}
$$
$\hat{L}_\mu(\varphi)$ coincides with $\hat{\mu}(x) = \int_{\mathbb{R}^n} e^{-2\pi i x \cdot \xi} \, d\mu(\xi)$.

(3) A measurable function $f$ satisfying $\frac{f(x)}{(1+|x|^2)^k} \in L^p$, where $1 \leqslant p \leqslant \infty$ and $k \in \mathbb{N}$, is called a tempered function. (When $p=\infty$, such a function is called a slowly increasing function.)
\begin{theorem}
    A linear functional $L$ on $S$ is a tempered distribution if and only if there exist constants $C > 0$ and integers $m$ and $l$ such that
$$
|L(\varphi)| \leqslant C \sum_{\substack{|\alpha| \leqslant l \\ |\beta| \leqslant m}} \|\varphi\|_{\alpha, \beta} \quad \forall \varphi \in S.
$$
\end{theorem}
~\\
\textbf{\large{Convolution of a distribution with a function in $S$}}

For a function $g$ on $\mathbb{R}^n$, its reflection $\tilde{g}$ is defined by $\tilde{g}(x) = g(-x)$. If $u, \varphi, \psi \in S$, then
$$
\int_{\mathbb{R}^n} (u * \varphi)(x) \psi(x) \, dx = \int_{\mathbb{R}^n} u(x) (\tilde{\varphi} * \psi)(x) \, dx.
$$
The mappings $\psi \longmapsto \int_{\mathbb{R}^n} (u * \varphi)(x) \psi(x) \, dx$ and $\theta \longmapsto \int_{\mathbb{R}^n} u(x) \theta(x) \, dx$ are linear functionals on $S$. Denote these functionals by $u * \varphi$ and $u$, respectively. Then, $(*)$ is given by
$$
(u * \varphi)(\psi) = u(\tilde{\varphi} * \psi).
$$
\begin{definition}
    Let $u \in S'$ and $\varphi \in S$. Define the convolution $u * \varphi$ by $(u * \varphi)(\psi) = u(\tilde{\varphi} * \psi)$. Then, for all $u \in S'$ and $\varphi \in S$, we have $u * \varphi \in S'$ and the convolution is associative: $(u * \varphi) * \psi = u * (\varphi * \psi)$ whenever $u \in S'$ and $\varphi, \psi \in S$.
\end{definition}
\begin{theorem}
    If $u \in S'$ and $\varphi \in S$, then the convolution $u * \varphi$ coincides with the function $f$ defined by $f(x) = u(\tau_x \tilde{\varphi})$ for $x \in \mathbb{R}^n$, where $\tau_x$ denotes the translation by $x$ (i.e., $\tau_x(g(y)) = g(y-x)$). Moreover, $f \in C^\infty$ and it as well as all its derivatives are slowly increasing, i.e., for all $\alpha$ there exist constants $C_\alpha, k_\alpha > 0$ such that
$$
\left| (\partial^\alpha f)(x) \right| \leqslant C_\alpha (1 + |x|)^{k_\alpha}.
$$
\end{theorem}
\begin{proof}
By the continuity of $u$ and the fact that
$$
\frac{\tau_{h e_j}(\tau_x(\tilde{\varphi})) - \tau_x(\tilde{\varphi})}{h} \rightarrow -\tau_x(\partial_j \tilde{\varphi})
$$
in $S$ as $h \rightarrow 0$, we have
    $$
    \frac{f(x+he_j)-f(x)}{h}=\frac{u(\tau_{x+he_j} \tilde{\phi})-u(\tau_x \tilde{\phi})}{h}=u(\frac{\tau_{he_j}( \tau_{x+he_j}\tilde{\phi})-\tau_x \tilde{\phi}}{h}) \rightarrow -u(T_x(\partial_j \tilde{\varphi})).
    $$

Considering higher-order derivatives, we find that the function $f$ belongs to the class $C^{\infty}$ and satisfies the relation:
$$
\partial^\alpha f(x) = (-1)^{|\alpha|} u(\tau_x D^\alpha \tilde{\varphi}).
$$
Moreover, we can estimate the magnitude of $\partial^\alpha f(x)$ as:
$$
\begin{aligned}
|\partial^\alpha f(x)| &\leq c \sum_{|\gamma| \leq l} \sup_{y \in \mathbb{R}^n} |y^\gamma \tau_x(\partial^{\alpha+\beta} \tilde{\varphi})(y)| \\
&\leq c \sum_{\substack{|\gamma| \leq l \\ |\beta| \leq m}} \sup_{y \in \mathbb{R}^n} |(x+y)^\gamma (\partial^{\alpha+\beta} \tilde{\varphi})(y)| \\
&\leq C_l \sum_{|\beta| \leq m} \sup_{y \in \mathbb{R}^n} (1 + |x|^l + |y|^l) |(\partial^{\alpha+\beta} \tilde{\varphi})(y)|.
\end{aligned}
$$
This estimation reveals that $|\partial^\alpha f(x)|$ is bounded by a polynomial of $x$.

Now, let's demonstrate that for any $\psi$ belonging to the Schwartz space $S$, the following equality holds:
$$
(u * \varphi)(\psi) = \int_{\mathbb{R}^n} f(x) \psi(x) \, dx.
$$
To this end, we observe that:
$$
\begin{aligned}
(u * \varphi)(\psi) &= u(\tilde{\varphi} * \psi) \\
&= u\left( \int_{\mathbb{R}^n} \tilde{\varphi}(x-y) \psi(y) \, dy \right) \\
&= u\left( \int_{\mathbb{R}^n} (\tau_y \tilde{\varphi})(x) \psi(y) \, dy \right) \\
&= \int_{\mathbb{R}^n} u(\tau_y \tilde{\varphi}) \psi(y) \, dy.
\end{aligned}
$$
Here, the transition from the Riemann sum of $\int_{\mathbb{R}^n} (\tau_y \tilde{\varphi})(x) \psi(y) \, dy$ to the integral is justified by the linearity and continuity of the functional $u$ in the Schwartz space $S$.
\end{proof}
\chapter{The Theory of Singular Integrals}
\section{The Hilbert transform: A model}
\begin{definition}
    The principal value of the function $\frac{1}{x}$ is denoted as $p.v. \frac{1}{x}$. We define $\omega_0 = p.v. \frac{1}{x}$ as follows: For $\varphi \in S(\mathbb{R})$ (functions in the Schwartz space), we have

$$\omega_0(\varphi) = \lim_{\varepsilon \rightarrow 0} \int_{|x| > \varepsilon} \frac{\varphi(x)}{x} \, dx$$
\end{definition}
\begin{proposition}
     $\omega_0 \in S'$, which means $\omega_0$ is a tempered distribution.
\end{proposition}
\begin{proof}
    $$\begin{aligned}
\omega_0(\varphi) &= \lim_{\varepsilon \rightarrow 0} \left( \int_{\varepsilon < |x| \leq 1} \frac{\varphi(x)}{x} \, dx + \int_{|x| \geq 1} \frac{\varphi(x)}{x} \, dx \right) \\
&= \lim_{\varepsilon \rightarrow 0} \left( \int_{\varepsilon < |x| \leq 1} \frac{\varphi(x) - \varphi(0)}{x} \, dx + \int_{|x| \geq 1} \frac{\varphi(x)}{x} \, dx \right)
\end{aligned}$$Since $\left| \frac{\varphi(x) - \varphi(0)}{x} \right| \leq \|\varphi'\|_\infty$, by the Dominated Convergence Theorem (DCT), we obtain

$$\begin{aligned}
|\omega_0(\varphi)| &\leq \int_{\mathbb{R}} \left| \frac{\varphi(x) - \varphi(0)}{x} \right| \, dx + \int_{|x| \geq 1} \left| \frac{\varphi(x)}{x} \right| \, dx \\
&\leq 2\|\varphi'\|_\infty + \int_{|x| \geq 1} \left| \frac{x \varphi(x)}{x^2} \right| \, dx \\
&\leq 2\|\varphi'\|_\infty + 2\|x \varphi(x)\|_\infty
\end{aligned}$$Therefore, $\omega_0 \in S'$.
\end{proof}
\begin{definition}
    For $f \in S(\mathbb{R})$, the Truncated Hilbert Transform (at height $\epsilon$) is defined as
    $$H^{(\varepsilon)}(f)(x) = \frac{1}{\pi} \int_{|y| \geq \varepsilon} \frac{f(x-y)}{y} \, dy = \frac{1}{\pi} \int_{|x-y| > \varepsilon} \frac{f(y)}{x-y} \, dy$$
    The Hilbert Transform is then defined as
    $$Hf(x) = \frac{1}{\pi} (w_0 * f)(x) = \lim_{\varepsilon \rightarrow 0} H^{(\varepsilon)}(f)(x)$$
\end{definition}
\noindent\textbf{Note}. The convolution $\omega_0 * f(x)$ can be expressed as $\omega_0(\tau_x \tilde{f})$, where $\tau_x \tilde{f}(y) = \tilde{f}(y-x)$, i.e.,

$$\omega_0 * f(x) = \lim_{\varepsilon \rightarrow 0} \int_{|y| > \varepsilon} \frac{\tau_x \tilde{f}(y)}{y} \, dy = \lim_{\varepsilon \rightarrow 0} \int_{|y| > \varepsilon} \frac{f(x-y)}{y} \, dy$$
Alternatively, the Hilbert Transform can also be represented as

$$Hf(x) = \frac{1}{\pi} p.v. \int_{-\infty}^{+\infty} \frac{f(x-y)}{y} \, dy = \frac{1}{\pi} p.v. \int_{-\infty}^{+\infty} \frac{f(y)}{x-y} \, dy$$
\noindent\textbf{Remark}. If we use $\lim_{\varepsilon \rightarrow 0} \int_{|y| > \varepsilon} \frac{f(x-y)}{y} \, dy$ to define the Hilbert Transform , its definition can be naturally extended to a broader class of functions. Given $x \in \mathbb{R}$, $Hf(x)$ is defined for all integrable functions $f$ on $\mathbb{R}$ that satisfy the Hölder condition near $x$, i.e., there exist $C_x, \varepsilon_x > 0$ such that $|f(x) - f(y)| \leq C_x |x-y|^{\varepsilon_x}$ whenever $|y-x| < \delta_x$.

For piecewise smooth integrable functions, the Hilbert Transform is well-defined at the Hölder-Lipschitz continuous points of the function.
\begin{proposition}
    $\left(\frac{1}{\pi} p.v. \frac{1}{x}\right)^{\wedge}(\xi) $ is coincide with $ -i \, \text{sgn}(\xi)$ i.e,  $$\left(\frac{1}{\pi} p.v. \frac{1}{x}\right)^{\wedge}(\varphi) = \int (-i \, \text{sgn}(\xi)) \varphi(\xi) \, d\xi$$.
\end{proposition}
\begin{proposition}
    $\widehat{Hf}(\xi) $ is coincide with $ -i \, \text{sgn}(\xi) \hat{f}(\xi)$ for $f \in S(\mathbb{R})$ i.e, for all $\varphi \in S(\mathbb{R})$, we have

$$\widehat{Hf}(\varphi) = \int_{\mathbb{R}} -i \, \text{sgn}(\xi) \hat{f}(\xi) \varphi(\xi) \, d\xi.$$
\end{proposition}
\begin{proof}
    In fact, $\widehat{Hf}(\varphi) = Hf(\hat{\varphi}) = \frac{1}{\pi}(\omega_0 * f)(\hat{\varphi})$ can be further derived as:

$$\begin{aligned}
&= \frac{1}{\pi} \omega_0(\tilde{f} * \hat{\varphi}) = \frac{1}{\pi} \omega_0(\widehat{\hat{f}}) \\
&= \frac{1}{\pi} \omega_0(\hat{f} \varphi) \\
&= \int_{\mathbb{R}} -i \, \text{sgn}(\xi) \hat{f}(\xi) \varphi(\xi) \, d\xi.
\end{aligned}$$
this is what we desired.
\end{proof}
We consider $\widehat{Hf}(\xi)$ as a function and identify $\hat{H} \in S'$ with the function $-i \, \text{sgn}(\xi) \hat{f}(\xi)$. Therefore, for $f \in S$, we have $\|\hat{Hf}\|_2 = \|\hat{f}\|_2$.

Using this isometric property, we can extend the definition of the Hilbert Transform to $L^2(\mathbb{R})$. If $f \in L^2(\mathbb{R})$, then $-i \, \text{sgn}(\xi) \hat{f}(\xi) \in L^2$. We define $Hf(x) = (-i \, \text{sgn}(\xi) \hat{f}(\xi))^\vee(x)$. If $f \in L^2$ and there exists a sequence $\{f_n\} \subset S$ converging to $f$ in $L^2$, then $\{\hat{Hf_m}\}$ is a Cauchy sequence in $L^2$ and thus converges to a function in $L^2$.
For $f$ in $L^2$, we can define its Hilbert transform via $g \in L^2$ where $\hat{g} = \lim_{{m \to \infty}} \widehat{H f_m}$ in $L^2$.

\begin{theorem}\label{thm3.1.6}
    For $f \in S(\mathbb{R})$,\\
(1) $H$ is of weak type $(1,1)$, i.e., $m\{x \in \mathbb{R}: |H f(x)| > \lambda\} \leqslant \frac{c}{\lambda} \|f\|_1$;\\
(2) $H$ is of strong type $(p, p)$ for $1 < p < \infty$, i.e., $\|H f\|_p \leqslant C_p \|f\|_p$.
\end{theorem} 
\noindent\textbf{Remark}. (1) As $p \to \infty$, $C_p = O(p)$; as $p \to 1$, $C_p = O\left(\frac{1}{p-1}\right)$;\\
(2) If $f = \chi_{[0,1]}$, then $H f(x) = \frac{1}{\pi} \log \left|\frac{x}{x-1}\right|$. Note that while $f \in L^1$, $H f \notin L^1$ and similarly, while $f \in L^{\infty}$, $H f \notin L^{\infty}$.
\begin{lemma}[The Calderon-Zygmund Decomposition in $L^{1}\left(\mathbb{R}^n\right)$]
    Let $f \in L^{\prime}\left(\mathbb{R}^n\right)$ and $\lambda > 0$. Then $f$ can be decomposed as $f = g + b$ where $|g| \leqslant \lambda$ a.e. and $b = \sum_Q x_Q f$. The summation is over a collection $B = \{Q\}$ of disjoint cubes, and for each $Q$, $\lambda < \frac{1}{|Q|} \int_Q |f(x)| dx \leqslant 2^n \lambda $ (\ding{172}). Furthermore, $m\left(\cup_{Q \in B} Q\right) < \frac{1}{\lambda} \|f\|_1$ (\ding{173}).
\end{lemma}
\begin{proof}
For each $l \in \mathbb{Z}$, define a collection of dyadic cubes $D_l$ as follows:
$$
D_l = \left\{ \prod_{i=1}^n \left[ 2^l m_i, 2^l (m_i + 1) \right) : m_1, \ldots, m_n \in \mathbb{Z} \right\}
$$
Observe that if $Q \in D_l$ and $Q' \in D_{l'}$, then either $Q \cap Q' = \emptyset$, $Q \subset Q'$, or $Q' \subset Q$.
Choose $l_0$ large enough so that for each $Q \in D_{l_0}$ satisfies $\frac{1}{|Q|} \int_Q |f(x)| dx \leq \lambda$.
For each such cube, consider its $2^n$ "children" (or subcube) with side length $2^{l_0 - 1}$.
Each subcube $Q'$ will have one of the following properties:
\begin{equation}\label{3.1}
    \frac{1}{|Q'|} \int_{Q'} |f(x)| dx \leqslant \lambda \quad \text{or} \quad \frac{1}{|Q'|} \int_{Q'} |f(x)| dx > \lambda
\end{equation}
In the latter case, we stop and include $Q'$ in the collection $B$.
Observe that in this case,
$$
\frac{1}{|Q'|} \int_{Q'} |f(x)| dx \leqslant \frac{2^n}{|Q|} \int_Q |f(x)| dx \leqslant 2^n \lambda
$$
Let $Q$ denote the parent cube of $Q'$. Therefore, (\ding{172}) holds. If the first inequality in \eqref{3.1} is satisfied, then further subdivide $Q'$ into its child cubes, each with half the side length of $Q'$. Continuing this process yields a collection of disjoint dyadic cubes $B$ that satisfy (\ding{172}). Consequently, (\ding{173}) also holds because
$$
\left|\bigcup_{Q \in B} Q\right| \leq \sum_{Q \in B}|Q| < \sum_{Q \in B} \frac{1}{\lambda} \int_Q |f(x)| dx = \frac{1}{\lambda} \int_{\bigcup Q} |f(x)| dx \leq \frac{1}{\lambda} \|f\|_1.
$$
Now, consider a point $x_0 \in \mathbb{R}^n \setminus \bigcup_{Q \in B} Q$. Such an $x_0$ is contained in a decreasing sequence of dyadic cubes $\left\{Q_i\right\}$, each satisfying $\frac{1}{\left|Q_j\right|} \int_{Q_j} |f| \leq \lambda$.
By Lebesgue's theorem, for such an $x_0$, we have $\left|f\left(x_0\right)\right| \leq \lambda$ almost everywhere.

Define $g = f - b = f - \sum_{Q \in B} x_Q f = 1_{\mathbb{R}^n \setminus \bigcup_{Q \in B}} f$. Since $\mathbb{R}^n \setminus \bigcup_{Q \in B} Q$ and $\mathbb{R}^n \setminus \bigcup_{Q \in \bar{Q}} Q$ differ only by a set of measure zero, it follows that $|g| \leq \lambda$ almost everywhere, as desired.
\end{proof}
\noindent\textbf{Proof of boundedness of $H$ (Theorem \ref{thm3.1.6})}:\\
(1) Fix $\lambda > 0$. Using the $C-Z$ decomposition, there exist disjoint intervals $\left\{I_j\right\}$ such that
$$
|f| \leq \lambda \text{ a.e. } x \notin \Omega = \bigcup_j I_j, \quad \lambda < \frac{1}{\left|I_j\right|} \int_{I_j} |f(x)| dx \leq 2\lambda, \quad |\Omega| \leq \frac{1}{\lambda} \|f\|_1.
$$
Decompose $f$ as $f = g + b$, where
$$
g(x) = \begin{cases} 
f(x) & x \notin \Omega \\
\frac{1}{\left|I_j\right|} \int_{I_j} f(x) dx & x \in I_j, j \in \mathbb{N}
\end{cases}
$$
and $b(x) = \sum_j b_j(x)$ with $b_j(x) = \left(f(x) - \frac{1}{\left|I_j\right|} \int_{I_j} f(x) dx\right) \textbf{1}_{I_j}(x)$. Then $|g(x)| \leq 2\lambda$ almost everywhere, and each $b_j$ is supported on $I_j$ with $\int_{I_j} b_j (x) dx = 0$.

Since $f = g + b$, we have $H f = H g + H b$. Consequently,
$$
\left|\left\{x: |H f(x)| > \lambda\right\}\right| \leq \left|\left\{x: |H g(x)| > \frac{\lambda}{2}\right\}\right| + \left|\left\{x: |H b(x)| > \frac{\lambda}{2}\right\}\right|.
$$
For the first term, we have
$$
\begin{aligned}
\left|\left\{x: |Hg(x)| > \frac{\lambda}{2}\right\}\right| &\leqslant \frac{1}{(\lambda/2)^2} \int |Hg(x)|^2 dx = \frac{4}{\lambda^2} \int |g(x)|^2 dx \\
&\leqslant \frac{4}{\lambda^2} \left( \int |g(x)| dx \right)^2 \leqslant \frac{4}{\lambda^2} \left( \int |f(x)| dx \right)^2 \\
&\leqslant \frac{4}{\lambda^2} \|f\|_1^2 < \infty \quad \text{(Since } f \in L^1 \subset L^2\text{)}
\end{aligned}
$$
Let $2I_j$ be the interval with the same center as $I_j$ and twice the length, and let $\Omega^* = \bigcup_j 2I_j$. Then $|\Omega^*| \leq 2|\Omega|$ and
$$
\begin{aligned}
\left|\left\{x: |Hb(x)| > \frac{\lambda}{2}\right\}\right| &\leq |\Omega^*| + \left|\left\{x \notin \Omega^*: |Hb(x)| > \frac{\lambda}{2}\right\}\right| \\
&\leq 2|\Omega| + \frac{2}{\lambda} \int_{\mathbb{R} \setminus \Omega^*}|Hb(x)| dx \\
&\leq \frac{2}{\lambda}\|f\|_1 + \frac{2}{\lambda} \int_{\mathbb{R} \setminus \Omega^*}|Hb(x)| dx
\end{aligned}
$$
Note that
$$
\begin{aligned}
\int_{\mathbb{R} \setminus \Omega^*}|Hb(x)| dx &\leq \int_{\mathbb{R} \setminus \Omega^*}\left|\sum_j Hb_j(x)\right| dx \leq \sum_j \int_{\mathbb{R} \setminus \Omega^*}|Hb_j(x)| dx \\
&\leq \sum_j \int_{\mathbb{R} \setminus 2I_j}|Hb_j(x)| dx
\end{aligned}
$$
and
$$
\begin{aligned}
\int_{\mathbb{R} \setminus 2I_j}|Hb_j(x)| dx &= \int_{\mathbb{R} \setminus 2I_j} \lim_{\varepsilon \rightarrow 0} \left| \int_{\substack{|x-y|>\varepsilon \\ y \in I_j}} \frac{b_j(y)}{x-y} dy \right| dx \\
&= \int_{\mathbb{R} \setminus 2I_j} \left| \int_{I_j} b_j(y) \left( \frac{1}{x-y} - \frac{1}{x-c_j} \right) dy \right| dx
\end{aligned}
$$
where $c_j$ is the center of $I_j$ and $\int_{\mathbb{R}} b_j(x) dx = 0$. Since $|y -c_j| \leq \frac{1}{2} |I_j|$ and $|x-y| \geq \frac{1}{2} |x-c_j|$, then,
$$
\begin{aligned}
\int_{\mathbb{R} \setminus 2I_j} \left| \int_{I_j} b_j(y) \left( \frac{1}{x-y} - \frac{1}{x-c_j} \right) dy \right| dx &\leq \int_{I_j} |b_j(y)| \int_{\mathbb{R} \setminus 2I_j} \frac{|y-c_j|}{|x-y||x-c_j|} dx dy \\
&\leq \int_{I_j} |b_j(y)| \int_{\mathbb{R} \setminus 2I_j} \frac{|I_j|}{|x-c_j|^2} dx dy \\
&= 2 \int_{I_j} |b_j(y)| dy
\end{aligned}
$$
Thus,
\[  
\sum_j \int_{\mathbb{R}^{2I_j}} |Hb_j(x)| \, dx \leqslant 2 \sum_j \int_{I_j} |b_j(x)| \, dx \leqslant 4 \sum_j \int_{I_j} |f(x)| \, dx \leqslant 4\|f\|_1.  
\]  
Then, we can show that  
\[  
\left|\left\{ x : |Hb(x)| > \frac{\lambda}{2} \right\}\right| \leq \frac{10}{\lambda}\|f\|_1,  
\]  
which proves the weak type $(1,1)$ estimate. \\ 
  
(2) $H$ is of weak type $(1,1)$ and strong type $(2,2)$ (since $\|Hf\|_2 = \|f\|_2$). Hence, by interpolation, it is also of strong type $(p,p)$ for $1 < p < 2$.  
If $2 < p < \infty$, then $p' < 2$. Consider the following estimate:  
\[  
\begin{aligned}  
\|Hf\|_p &= \sup \left\{ \left| \int_{\mathbb{R}} Hf \cdot g \, dx \right| : g \in C_c^{\infty}, \|g\|_{p'} \leq 1 \right\} \\  
&= \sup \left\{ \left| -\int f \cdot Hg \right| : g \in C_c^{\infty}, \|g\|_{p'} \leq 1 \right\}.  
\end{aligned}  
\]  
For $f , g \in S(\mathbb{R})$, we have  
\[  
\int Hf \cdot g = \int Hf(\tilde{g})^{\wedge} = \int \hat{H}_f \tilde{g} = \int -i \operatorname{sgn}(\xi) \hat{f}(\xi) \hat{g}(\xi) \, d\xi.  
\]  
Furthermore, by a change of variables ($\eta = -\xi$), we obtain  
\[  
\begin{aligned}  
&\phantom{{}={}} \int -i \operatorname{sgn}(\xi) \hat{f}(\xi) \hat{g}(\xi) \, d\xi \\  
&= -\int i \operatorname{sgn}(\eta) \hat{f}(\eta) \hat{g}(\eta) \, d\eta \\  
&= -\int \hat{f}(\eta) H\hat{g}(\eta) \, d\eta \\  
&= -\int (\hat{f})^{\wedge} Hg \\  
&= -\int f \cdot Hg.  
\end{aligned}  
\]  
Using Holder's Inequality, we have  
\[  
\|Hf\|_p \leqslant C\|f\|_p \sup \left\{ \|g\|_{p'} : g \in C_c^{\infty}, \|g\|_{p'} \leq 1 \right\} \leqslant C\|f\|_p.  
\]  
\noindent\textbf{Remark}. We can extend the Hilbert transform to functions in $L^p$ space. For $1 \leq p < \infty$ and any $f \in L^p$, there exists a sequence $\{f_k\} \subset S$ such that $f_k \rightarrow f$ in $L^p$. Since  
\[  
\|Hf_m - Hf_n\|_p \leq C\|f_m - f_n\|_p \rightarrow 0  
\]  
as $m, n \rightarrow \infty$, the sequence $\{Hf_m\}$ is Cauchy in $L^p$ and converges to some $g = Hf$.

Another similar example is the \textbf{Riesz transform}
\begin{definition}
    For $\varphi \in \mathcal{S}(\mathbb{R}^n)$ and $w_j \in \mathcal{S}'$, let
\[
\left\langle w_j, \varphi \right\rangle = \frac{\Gamma\left(\frac{n+1}{2}\right)}{\pi^{\frac{n+1}{2}}} \lim_{\varepsilon \rightarrow 0} \int_{|y| \geqslant \varepsilon} \frac{y_j}{|y|^{n+1}} \varphi(y) \, dy.
\]
and for $1 \leq j \leq n$, the \textbf{j-th Riesz transform} of $f$ is given by
\[
R_j(f)(x) = (w_j * f)(x) = \frac{\Gamma\left(\frac{n+1}{2}\right)}{\pi^{\frac{n+1}{2}}} \, \text{p.v.} \int_{\mathbb{R}^n} \frac{x_j-y_j}{|x-y|^{n+1}} f(y) \, dy, \quad \forall f \in \mathcal{S}(\mathbb{R}^n),
\]
where p.v. denotes the principal value integral.
\end{definition}
\noindent\textbf{Remark}. The definition makes sense for any integrable function that satisfies the following property: for all $x$, there exist constants $C_x > 0$, $\varepsilon_x > 0$, and $\delta_x > 0$ such that
\[
|f(x) - f(y)| \leqslant C_x |x-y|^{\varepsilon_x}
\]
whenever $|x-y| < \delta_x$.
\section{Singular Integrals}
\begin{definition}
    Let $K: \mathbb{R}^n \setminus \{0\} \rightarrow \mathbb{C}$ satisfy, for some constant $B$,
\begin{enumerate}
    \item $|K(x)| \leqslant B|x|^{-n}$ for $x \in \mathbb{R}^n \setminus \{0\}$ \quad (size condition)
    \item $\int_{|x|>2|y|}|K(x) - K(x-y)| \, dx \leq B$ for all $y$ \quad (smoothness condition)
    \item $\int_{r<|x|<S} K(x) \, dx = 0$ for all $0 < r < S < \infty$ \quad (cancellation condition)
\end{enumerate}
Then $K$ is called a Calder\'{o}n-Zygmund kernel.

The singular integral operator (or Calder\'{o}n-Zygmund operator) with kernel $K$ is defined as
\[
Tf(x) = \lim_{\varepsilon \rightarrow 0} \int_{|x-y|>\varepsilon} K(x-y) f(y) \, dy, \quad \forall f \in \mathcal{S}(\mathbb{R}^n).
\]
\end{definition}
\begin{lemma}
    Suppose that $|\nabla K(x)| \leq B|x|^{-n-1}$ for all $x \neq 0$ and some $B > 0$. Then
\[
\int_{|x|>2|y|}|K(x) - K(x-y)| \, dx \leq CB
\]
with $C = C(n)$ being a constant depending only on $n$.
\end{lemma}
\end{document}
